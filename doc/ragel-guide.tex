%
%   Copyright 2001-2007 Adrian Thurston <thurston@cs.queensu.ca>
%

%   This file is part of Ragel.
%
%   Ragel is free software; you can redistribute it and/or modify
%   it under the terms of the GNU General Public License as published by
%   the Free Software Foundation; either version 2 of the License, or
%   (at your option) any later version.
%
%   Ragel is distributed in the hope that it will be useful,
%   but WITHOUT ANY WARRANTY; without even the implied warranty of
%   MERCHANTABILITY or FITNESS FOR A PARTICULAR PURPOSE.  See the
%   GNU General Public License for more details.
%
%   You should have received a copy of the GNU General Public License
%   along with Ragel; if not, write to the Free Software
%   Foundation, Inc., 59 Temple Place, Suite 330, Boston, MA  02111-1307  USA 

% TODO: Need a section on the different strategies for handline recursion.

\documentclass[letterpaper,11pt,oneside]{book}
\usepackage{graphicx}
\usepackage{comment}
\usepackage{multicol}

\topmargin -0.20in
\oddsidemargin 0in
\textwidth 6.5in
\textheight 9in

\setlength{\parskip}{0pt}
\setlength{\topsep}{0pt}
\setlength{\partopsep}{0pt}
\setlength{\itemsep}{0pt}

\input{version}

\newcommand{\verbspace}{\vspace{10pt}}
\newcommand{\graphspace}{\vspace{10pt}}

\renewcommand\floatpagefraction{.99}
\renewcommand\topfraction{.99}
\renewcommand\bottomfraction{.99}
\renewcommand\textfraction{.01}   
\setcounter{totalnumber}{50}
\setcounter{topnumber}{50}
\setcounter{bottomnumber}{50}

\newenvironment{inline_code}{\def\baselinestretch{1}\vspace{12pt}\small}{}

\begin{document}

%
% Title page
%
\thispagestyle{empty}
\begin{center}
\vspace*{3in}
{\huge Ragel State Machine Compiler}\\
\vspace*{12pt}
{\Large User Guide}\\
\vspace{1in}
by\\
\vspace{12pt}
{\large Adrian Thurston}\\
\end{center}
\clearpage

\pagenumbering{roman}

%
% License page
%
\chapter*{License}
Ragel version \version, \pubdate\\
Copyright \copyright\ 2003-2007 Adrian Thurston
\vspace{6mm}

{\bf\it\noindent This document is part of Ragel, and as such, this document is
released under the terms of the GNU General Public License as published by the
Free Software Foundation; either version 2 of the License, or (at your option)
any later version.}

\vspace{5pt}

{\bf\it\noindent Ragel is distributed in the hope that it will be useful, but
WITHOUT ANY WARRANTY; without even the implied warranty of MERCHANTABILITY or
FITNESS FOR A PARTICULAR PURPOSE.  See the GNU General Public License for more
details.}

\vspace{5pt}

{\bf\it\noindent You should have received a copy of the GNU General Public
License along with Ragel; if not, write to the Free Software Foundation, Inc.,
59 Temple Place, Suite 330, Boston, MA  02111-1307  USA}

%
% Table of contents
%
\clearpage
\tableofcontents
\clearpage

%
% Chapter 1
%

\pagenumbering{arabic}

\chapter{Introduction}

\section{Abstract}

Regular expressions are used heavily in practice for the purpose of specifying
parsers. However, they are normally used as black boxes linked together with
program logic.  User actions are executed in between invocations of the regular
expression engine. Adding actions before a pattern terminates requires patterns
to be broken and pasted back together with program logic. The more user actions
are needed, the less the advantages of regular expressions are seen. 

Ragel is a software development tool that allows user actions to be 
embedded into the transitions of a regular expression's corresponding state
machine, eliminating the need to switch from the regular expression engine and
user code execution environment and back again. As a result, expressions can be
maximally continuous.  One is free to specify an entire parser using a single
regular expression.  The single-expression model affords concise and elegant
descriptions of languages and the generation of very simple, fast and robust
code.  Ragel compiles finite state machines from a high level regular language
notation to executable C, C++, Objective-C, D, Java or Ruby.

In addition to building state machines from regular expressions, Ragel allows
the programmer to directly specify state machines with state charts. These two
notations may be freely combined. There are also facilities for controlling
nondeterminism in the resulting machines and building scanners using patterns
that themselves have embedded actions. Ragel can produce code that is small and
runs very fast. Ragel can handle integer-sized alphabets and can compile very
large state machines.

\section{Motivation}

When a programmer is faced with the task of producing a parser for a
context-free language there are many tools to choose from. It is quite common
to generate useful and efficient parsers for programming languages from a
formal grammar. It is also quite common for programmers to avoid such tools
when making parsers for simple computer languages, such as file formats and
communication protocols.  Such languages often meet the criteria for the
regular languages.  Tools for processing the context-free languages are viewed
as too heavyweight for the purpose of parsing regular languages because the extra
run-time effort required for supporting the recursive nature of context-free
languages is wasted.

When we turn to the regular expression-based parsing tools, such as Lex, Re2C,
and scripting languages such as Sed, Awk and Perl we find that they are split
into two levels: a regular expression matching engine and some kind of program
logic for linking patterns together.  For example, a Lex program is composed of
sets of regular expressions. The implied program logic repeatedly attempts to
match a pattern in the current set, then executes the associated user code. It requires the
user to consider a language as a sequence of independent tokens.  Scripting
languages and regular expression libraries allow one to link patterns together
using arbitrary program code.  This is very flexible and powerful, however we
can be more concise and clear if we avoid gluing together regular expressions
with if statements and while loops.

This model of execution, where the runtime alternates between regular
expression matching and user code exectution places severe restrictions on when
action code may be executed. Since action code can only be associated with
complete patterns, any action code that must be executed before an entire
pattern is matched requires that the pattern be broken into smaller units.
Instead of being forced to disrupt the regular expression syntax and write
smaller expressions, it is desirable to retain a single expression and embed
code for performing actions directly into the transitions that move over the
characters. After all, capable programmers are astutely aware of the machinery
underlying their programs, so why not provide them with access to that
machinery? To achieve this we require an action execution model for associating
code with the sub-expressions of a regular expression in a way that does not
disrupt its syntax.

The primary goal of Ragel is to provide developers with an ability to embed
actions into the transitions and states of a regular expression's state machine
in support of the
definition of entire parsers or large sections of parsers using a single
regular expression.  From the
regular expression we gain a clear and concise statement of our language. From
the state machine we obtain a very fast and robust executable that lends itself
to many kinds of analysis and visualization.

\section{Overview}

Ragel is a language for specifying state machines. The Ragel program is a
compiler that assembles a state machine definition to executable code.  Ragel
is based on the principle that any regular language can be converted to a
deterministic finite state automaton. Since every regular language has a state
machine representation and vice versa, the terms regular language and state
machine (or just machine) will be used interchangeably in this document.

Ragel outputs machines to C, C++, Objective-C, D, Java or Ruby code. The output is
designed to be generic and is not bound to any particular input or processing
method. A Ragel machine expects to have data passed to it in buffer blocks.
When there is no more input, the machine can be queried for acceptance.  In
this way, a Ragel machine can be used to simply recognize a regular language
like a regular expression library. By embedding code into the regular language,
a Ragel machine can also be used to parse input.

The Ragel language has many operators for constructing and manipulating
machines. Machines are built up from smaller machines, to bigger ones, to the
final machine representing the language that needs to be recognized or parsed.

The core state machine construction operators are those found in most theory
of computation textbooks. They date back to the 1950s and are widely studied.
They are based on set operations and permit one to think of languages as a set
of strings. They are Union, Intersection, Difference, Concatenation and Kleene
Star. Put together, these operators make up what most people know as regular
expressions. Ragel also provides a scanner construction operator 
and provides operators for explicitly constructing machines
using a state chart method. In the state chart method, one joins machines
together without any implied transitions and then explicitly specifies where
epsilon transitions should be drawn.

The state machine manipulation operators are specific to Ragel. They allow the
programmer to access the states and transitions of regular language's
corresponding machine. There are two uses of the manipulation operators. The
first and primary use is to embed code into transitions and states, allowing
the programmer to specify the actions of the state machine.

Ragel attempts to make the action embedding facility as intuitive as possible.
To do so, a number of issues need to be addressed.  For example, when making a
nondeterministic specification into a DFA using machines that have embedded
actions, new transitions are often made that have the combined actions of
several source transitions. Ragel ensures that multiple actions associated with
a single transition are ordered consistently with respect to the order of
reference and the natural ordering implied by the construction operators.

The second use of the manipulation operators is to assign priorities in
transitions. Priorities provide a convenient way of controlling any
nondeterminism introduced by the construction operators. Suppose two
transitions leave from the same state and go to distinct target states on the
same character. If these transitions are assigned conflicting priorities, then
during the determinization process the transition with the higher priority will
take precedence over the transition with the lower priority. The lower priority
transition gets abandoned. The transitions would otherwise be combined into a new
transition that goes to a new state that is a combination of the original
target states. Priorities are often required for segmenting machines. The most
common uses of priorities have been encoded into a set of simple operators
that should be used instead of priority embeddings whenever possible.

For the purposes of embedding, Ragel divides transitions and states into
different classes. There are four operators for embedding actions and
priorities into the transitions of a state machine. It is possible to embed
into start transitions, finishing transitions, all transitions and pending out
transitions.  The embedding of pending out transitions is a special case.
These transition embeddings get stored in the final states of a machine.  They
are transferred to any transitions that may be made going out of the machine by
a concatenation or kleene star operator.

There are several more operators for embedding actions into states. Like the
transition embeddings, there are various different classes of states that the
embedding operators access. For example, one can access start states, final
states or all states, among others. Unlike the transition embeddings, there are
several different types of state action embeddings. These are executed at
various different times during the processing of input. It is possible to embed
actions which are exectued on all transitions that enter into a state, all
transitions out of a state, transitions taken on the error event, or
transitions taken on the EOF event.

Within actions, it is possible to influence the behaviour of the state machine.
The user can write action code that jumps or calls to another portion of the
machine, changes the current character being processed, or breaks out of the
processing loop. With the state machine calling feature Ragel can be used to
parse languages that are not regular. For example, one can parse balanced
parentheses by calling into a parser when an open bracket character is seen and
returning to the state on the top of the stack when the corresponding closing
bracket character is seen. More complicated context-free languages such as
expressions in C, are out of the scope of Ragel. 

Ragel also provides a scanner construction operator which can be used to build scanners
much the same way that Lex is used. The Ragel generated code, which relies on
user-defined variables for
backtracking, repeatedly tries to match patterns to the input, favouring longer
patterns over shorter ones and patterns that appear ahead of others when the
lengths of the possible matches are identical. When a pattern is matched the
associated action is executed. 

The key distinguishing feature between scanners in Ragel and scanners in Lex is
that Ragel patterns may be arbitrary Ragel expressions and can therefore
contain embedded code. With a Ragel-based scanner the user need not wait until
the end of a pattern before user code can be executed.

Scanners do take Ragel out of the domain of pure state machines and require the
user to maintain the backtracking related variables.  However, scanners
integrate well with regular state machine instantiations. They can be called to
or jumped to only when needed, or they can be called out of or jumped out of
when a simpler, pure state machine model is appropriate.

Two types of output code style are available. Ragel can produce a table-driven
machine or a directly executable machine. The directly executable machine is
much faster than the table-driven. On the other hand, the table-driven machine
is more compact and less demanding on the host language compiler. It is better
suited to compiling large state machines.

\section{Related Work}

Lex is perhaps the best-known tool for constructing parsers from regular
expressions. In the Lex processing model, generated code attempts to match one
of the user's regular expression patterns, favouring longer matches over
shorter ones. Once a match is made it then executes the code associated with
the pattern and consumes the matching string.  This process is repeated until
the input is fully consumed. 

Through the use of start conditions, related sets of patterns may be defined.
The active set may be changed at any time.  This allows the user to define
different lexical regions. It also allows the user to link patterns together by
requiring that some patterns come before others.  This is quite like a
concatenation operation. However, use of Lex for languages that require a
considerable amount of pattern concatenation is inappropriate. In such cases a
Lex program deteriorates into a manually specified state machine, where start
conditions define the states and pattern actions define the transitions.  Lex
is therefore best suited to parsing tasks where the language to be parsed can
be described in terms of regions of tokens. 

Lex is useful in many scenarios and has undoubtedly stood the test of time.
There are, however, several drawbacks to using Lex.  Lex can impose too much
overhead for parsing applications where buffering is not required because all
the characters are available in a single string.  In these cases there is
structure to the language to be parsed and a parser specification tool can
help, but employing a heavyweight processing loop that imposes a stream
``pull'' model and dynamic input buffer allocation is inappropriate.  An
example of this kind of scenario is the conversion of floating point numbers
contained in a string to their corresponding numerical values.

Another drawback is the very issue that Ragel attempts to solve.
It is not possible to execute a user action while
matching a character contained inside a pattern. For example, if scanning a
programming language and string literals can contain newlines which must be
counted, a Lex user must break up a string literal pattern so as to associate
an action with newlines. This forces the definition of a new start condition.
Alternatively the user can reprocess the text of the matched string literal to
count newlines. 

\begin{comment}
How ragel is different from Lex.

%Like Re2c, Ragel provides a simple execution model that does not make any
%assumptions as to how the input is collected.  Also, Ragel does not do any
%buffering in the generated code. Consequently there are no dependencies on
%external functions such as \verb|malloc|. 

%If buffering is required it can be manually implemented by embedding actions
%that copy the current character to a buffer, or data can be passed to the
%parser using known block boundaries. If the longest-match operator is used,
%Ragel requires the user to ensure that the ending portion of the input buffer
%is preserved when the buffer is exhaused before a token is fully matched. The
%user should move the token prefix to a new memory location, such as back to the
%beginning of the input buffer, then place the subsequently read input
%immediately after the prefix.

%These properties of Ragel make it more work to write a program that requires
%the longest-match operator or buffering of input, however they make Ragel a
%more flexible tool that can produce very simple and fast-running programs under
%a variety of input acquisition arrangements.

%In Ragel, it is not necessary
%to introduce start conditions to concatenate tokens and retain action
%execution. Ragel allows one to structure a parser as a series of tokens, but
%does not require it.

%Like Lex and Re2C, Ragel is able to process input using a longest-match
%execution model, however the core of the Ragel language specifies parsers at a
%much lower level. This core is built around a pure state machine model. When
%building basic machines there is no implied algorithm for processing input
%other than to move from state to state on the transitions of the machine. This
%core of pure state machine operations makes Ragel well suited to handling
%parsing problems not based on token scanning. Should one need to use a
%longest-match model, the functionality is available and the lower level state
%machine construction facilities can be used to specify the patterns of a
%longest-match machine.

%This is not possible in Ragel. One can only program
%a longest-match instantiation with a fixed set of rules. One can jump to
%another longest-match machine that employs the same machine definitions in the
%construction of its rules, however no states will be shared.

%In Ragel, input may be re-parsed using a
%different machine, but since the action to be executed is associated with
%transitions of the compiled state machine, the longest-match construction does
%not permit a single rule to be excluded from the active set. It cannot be done
%ahead of time nor in the excluded rule's action.
\end{comment}

The Re2C program defines an input processing model similar to that of Lex.
Re2C focuses on making generated state machines run very fast and
integrate easily into any program, free of dependencies.  Re2C generates
directly executable code and is able to claim that generated parsers run nearly
as fast as their hand-coded equivalents.  This is very important for user
adoption, as programmers are reluctant to use a tool when a faster alternative
exists.  A consideration to ease of use is also important because developers
need the freedom to integrate the generated code as they see fit. 

Many scripting languages provide ways of composing parsers by linking regular
expressions using program logic. For example, Sed and Awk are two established
Unix scripting tools that allow the programmer to exploit regular expressions
for the purpose of locating and extracting text of interest. High-level
programming languages such as Perl, Python, PHP and Ruby all provide regular
expression libraries that allow the user to combine regular expressions with
arbitrary code.

In addition to supporting the linking of regular expressions with arbitrary
program logic, the Perl programming language permits the embedding of code into
regular expressions. Perl embeddings do not translate into the embedding of
code into deterministic state machines. Perl regular expressions are in fact
not fully compiled to deterministic machines when embedded code is involved.
They are instead interpreted and involve backtracking. This is shown by the
following Perl program. When it is fed the input \verb|abcd| the interpretor
attempts to match the first alternative, printing \verb|a1 b1|.  When this
possibility fails it backtracks and tries the second possibility, printing
\verb|a2 b2|, at which point it succeeds.

\begin{inline_code}
\begin{verbatim}
print "YES\n" if ( <STDIN> =~
        /( a (?{ print "a1 "; }) b (?{ print "b1 "; }) cX ) |
         ( a (?{ print "a2 "; }) b (?{ print "b2 "; }) cd )/x )
\end{verbatim}
\end{inline_code}
\verbspace

In Ragel there is no regular expression interpretor. Aside from the scanner
operator, all Ragel expressions are made into deterministic machines and the
run time simply moves from state to state as it consumes input. An equivalent
parser expressed in Ragel would attempt both of the alternatives concurrently,
printing \verb|a1 a2 b1 b2|.

\section{Development Status}

Ragel is a relatively new tool and is under continuous development. As a rough
release guide, minor revision number changes are for implementation
improvements and feature additions. Major revision number changes are for
implementation and language changes that do not preserve backwards
compatibility. Though in the past this has not always held true: changes that
break code have crept into minor version number changes. Typically, the
documentation lags behind the development in the interest of documenting only
the lasting features. The latest changes are always documented in the ChangeLog
file. 

\chapter{Constructing State Machines}

\section{Ragel State Machine Specifications}

A Ragel input file consists of a host language code file with embedded machine
specifications.  Ragel normally passes input straight to output.  When it sees
a machine specification it stops to read the Ragel statements and possibly generate
code in place of the specification.
Afterwards it continues to pass input through.  There
can be any number of FSM specifications in an input file. A multi-line FSM spec
starts with \verb|%%{| and ends with \verb|}%%|. A single-line FSM spec starts
with \verb|%%| and ends at the first newline.  

While Ragel is looking for FSM specifications it does basic lexical analysis on
the surrounding input. It interprets literal strings and comments so a
\verb|%%| sequence in either of those will not trigger the parsing of an FSM
specification. Ragel does not pass the input through any preprocessor nor does it
interpret preprocessor directives itself so includes, defines and ifdef logic
cannot be used to alter the parse of a Ragel input file. It is therefore not
possible to use an \verb|#if 0| directive to comment out a machine as is
commonly done in C code. As an alternative, a machine can be prevented from
causing any generated output by commenting out the write statements.

In Figure \ref{cmd-line-parsing}, a multi-line machine is used to define the
machine and single line machines are used to trigger the writing of the machine
data and execution code.

\begin{figure}
\begin{multicols}{2}
\small
\begin{verbatim}
#include <string.h>
#include <stdio.h>

%%{ 
    machine foo;
    main := 
        ( 'foo' | 'bar' ) 
        0 @{ res = 1; };
}%%

%% write data;
\end{verbatim}
\columnbreak
\begin{verbatim}
int main( int argc, char **argv )
{
    int cs, res = 0;
    if ( argc > 1 ) {
        char *p = argv[1];
        char *pe = p + strlen(p) + 1;
        %% write init;
        %% write exec;
    }
    printf("result = %i\n", res );
    return 0;
}
\end{verbatim}
\end{multicols}
\caption{Parsing a command line argument.}
\label{cmd-line-parsing}
\end{figure}

\subsection{Naming Ragel Blocks}

\begin{verbatim}
machine fsm_name;
\end{verbatim}
\verbspace

The \verb|machine| statement gives the name of the FSM. If present in a
specification, this statement must appear first. If a machine specification
does not have a name then Ragel uses the previous specification name.  If no
previous specification name exists then this is an error. Because FSM
specifications persist in memory, a machine's statements can be spread across
multiple machine specifications.  This allows one to break up a machine across
several files or draw in statements that are common to multiple machines using
the \verb|include| statement.

\subsection{Machine Definition}
\label{definition}

\begin{verbatim}
<name> = <expression>;
\end{verbatim}
\verbspace

The machine definition statement associates an FSM expression with a name.  Machine
expressions assigned to names can later be referenced by other expressions.  A
definition statement on its own does not cause any states to be generated. It is simply a
description of a machine to be used later. States are generated only when a definition is
instantiated, which happens when a definition is referenced in an instantiated
expression. 

\subsection{Machine Instantiation}
\label{instantiation}

\begin{verbatim}
<name> := <expression>;
\end{verbatim}
\verbspace

The machine instantiation statement generates a set of states representing an
expression. Each instantiation generates a distinct set of states.  The entry
point is written in the generated code using the instantiation name.  If the
\verb|main| machine is instantiated, its start state is used as the
specification's start state and is assigned to the \verb|cs| variable by the
\verb|write init| command. If no \verb|main| machine is given, the start state
of the last machine instantiation is used as the specification's start state.

From outside the execution loop, control may be passed to any machine by
assigning the entry point to the \verb|cs| variable.  From inside the execution
loop, control may be passed to any machine instantiation using \verb|fcall|,
\verb|fgoto| or \verb|fnext| statements.

\subsection{Including Ragel Code}

\begin{verbatim}
include FsmName "inputfile.rl";
\end{verbatim}
\verbspace

The \verb|include| statement can be used to draw in the statements of another FSM
specification. Both the name and input file are optional, however at least one
must be given. Without an FSM name, the given input file is searched for an FSM
of the same name as the current specification. Without an input file the
current file is searched for a machine of the given name. If both are present,
the given input file is searched for a machine of the given name.

\subsection{Importing Definitions}
\label{import}

\begin{verbatim}
import "inputfile.h";
\end{verbatim}
\verbspace

The \verb|import| statement takes a literal string as an argument, interprets
it as a file name, then scrapes the file for sequences of tokens that match the
following forms. If the input file is a Ragel program then tokens inside the
Ragel sections are ignored. See Section \ref{export} for a description of
exporting machine definitions.

\begin{itemize}
	\setlength{\itemsep}{-2mm}
    \item \verb|name = number|
    \item \verb|name = lit_string|
    \item \verb|"define" name number|
    \item \verb|"define" name lit_string|
\end{itemize}


\section{Lexical Analysis of a Ragel Block}
\label{lexing}

Within a machine specification the following lexical rules apply to the parse
of the input.

\begin{itemize}

\item The \verb|#| symbol begins a comment that terminates at the next newline.

\item The symbols \verb|""|, \verb|''|, \verb|//|, \verb|[]| behave as the
delimiters of literal strings. With them, the following escape sequences are interpreted: 

\verb|    \0 \a \b \t \n \v \f \r|

A backslash at the end of a line joins the following line onto the current. A
backslash preceding any other character removes special meaning. This applies
to terminating characters and to special characters in regular expression
literals. As an exception, regular expression literals do not support escape
sequences as the operands of a range within a list. See the bullet on regular
expressions in Section \ref{basic}.

\item The symbols \verb|{}| delimit a block of host language code that will be
embedded into the machine as an action.  Within the block of host language
code, basic lexical analysis of C/C++ comments and strings is done in order to
correctly find the closing brace of the block. With the exception of FSM
commands embedded in code blocks, the entire block is preserved as is for
identical reproduction in the output code.

\item The pattern \verb|[+-]?[0-9]+| denotes an integer in decimal format.
Integers used for specifying machines may be negative only if the alphabet type
is signed. Integers used for specifying priorities may be positive or negative.

\item The pattern \verb|0x[0-9A-Fa-f]+| denotes an integer in hexadecimal
format.

\item The keywords are \verb|access|, \verb|action|, \verb|alphtype|,
\verb|getkey|, \verb|write|, \verb|machine| and \verb|include|.

\item The pattern \verb|[a-zA-Z_][a-zA-Z_0-9]*| denotes an identifier.

%\item The allowable symbols are:
%
%\verb/    ( ) ! ^ * ? + : -> - | & . , := = ; > @ $ % /\\
%\verb|    >/  $/  %/  </  @/  <>/ >!  $!  %!  <!  @!  <>!|\\
%\verb|    >^  $^  %^  <^  @^  <>^ >~  $~  %~  <~  @~  <>~|\\
%\verb|    >*  $*  %*  <*  @*  <>*|

\item Any amount of whitespace may separate tokens.

\end{itemize}

%\section{Parse of an FSM Specification}

%The following statements are possible within an FSM specification. The
%requirements for trailing semicolons loosely follow that of C. 
%A block
%specifying code does not require a trailing semicolon. An expression
%statement does require a trailing semicolon.


\section{Basic Machines}
\label{basic}

The basic machines are the base operands of regular language expressions. They
are the smallest unit to which machine construction and manipulation operators
can be applied.

In the diagrams that follow the symbol \verb|df| represents
the default transition, which is taken if no other transition can be taken. The
symbol \verb|cr| represents the carriage return character, \verb|nl| represents the newline character (aka line feed) and the symbol
\verb|sp| represents the space character.

\begin{itemize}

\item \verb|'hello'| -- Concatenation Literal. Produces a machine that matches
the sequence of characters in the quoted string. If there are 5 characters
there will be 6 states chained together with the characters in the string. See
Section \ref{lexing} for information on valid escape sequences. 

% GENERATE: bmconcat
% OPT: -p
% %%{
% machine bmconcat;
\begin{comment}
\begin{verbatim}
main := 'hello';
\end{verbatim}
\end{comment}
% }%%
% END GENERATE

\begin{center}
\includegraphics[scale=0.55]{bmconcat}
\end{center}

It is possible
to make a concatenation literal case-insensitive by appending an \verb|i| to
the string, for example \verb|'cmd'i|.

\item \verb|"hello"| -- Identical to the single quoted version.

\item \verb|[hello]| -- Or Expression. Produces a union of characters.  There
will be two states with a transition for each unique character between the two states.
The \verb|[]| delimiters behave like the quotes of a literal string. For example, 
\verb|[ \t]| means tab or space. The \verb|or| expression supports character ranges
with the \verb|-| symbol as a separator. The meaning of the union can be negated
using an initial \verb|^| character as in standard regular expressions. 
See Section \ref{lexing} for information on valid escape sequences
in \verb|or| expressions.

% GENERATE: bmor
% OPT: -p
% %%{
% machine bmor;
\begin{comment}
\begin{verbatim}
main := [hello];
\end{verbatim}
\end{comment}
% }%%
% END GENERATE

\begin{center}
\includegraphics[scale=0.55]{bmor}
\end{center}

\item \verb|''|, \verb|""|, and \verb|[]| -- Zero Length Machine.  Produces a machine
that matches the zero length string. Zero length machines have one state that is both
a start state and a final state.

% GENERATE: bmnull
% OPT: -p
% %%{
% machine bmnull;
\begin{comment}
\begin{verbatim}
main := '';
\end{verbatim}
\end{comment}
% }%%
% END GENERATE

\begin{center}
\includegraphics[scale=0.55]{bmnull}
\end{center}

% FIXME: More on the range of values here.
\item \verb|42| -- Numerical Literal. Produces a two state machine with one
transition on the given number. The number may be in decimal or hexadecimal
format and should be in the range allowed by the alphabet type. The minimum and
maximum values permitted are defined by the host machine that Ragel is compiled
on. For example, numbers in a \verb|short| alphabet on an i386 machine should
be in the range \verb|-32768| to \verb|32767|.

% GENERATE: bmnum
% %%{
% machine bmnum;
\begin{comment}
\begin{verbatim}
main := 42;
\end{verbatim}
\end{comment}
% }%%
% END GENERATE

\begin{center}
\includegraphics[scale=0.55]{bmnum}
\end{center}

\item \verb|/simple_regex/| -- Regular Expression. Regular expressions are
parsed as a series of expressions that will be concatenated together. Each
concatenated expression
may be a literal character, the any character specified by the \verb|.|
symbol, or a union of characters specified by the \verb|[]| delimiters. If the
first character of a union is \verb|^| then it matches any character not in the
list. Within a union, a range of characters can be given by separating the first
and last characters of the range with the \verb|-| symbol. Each
concatenated machine may have repetition specified by following it with the
\verb|*| symbol. The standard escape sequences described in Section
\ref{lexing} are supported everywhere in regular expressions except as the
operands of a range within in a list. This notation also supports the \verb|i|
trailing option. Use it to produce case-insensitive machines, as in \verb|/GET/i|.

Ragel does not support very complex regular expressions because the desired
results can always be achieved using the more general machine construction
operators listed in Section \ref{machconst}. The following diagram shows the
result of compiling \verb|/ab*[c-z].*[123]/|.

% GENERATE: bmregex
% OPT: -p
% %%{
% machine bmregex;
\begin{comment}
\begin{verbatim}
main := /ab*[c-z].*[123]/;
\end{verbatim}
\end{comment}
% }%%
% END GENERATE

\begin{center}
\includegraphics[scale=0.55]{bmregex}
\end{center}

\item \verb|'a' .. 'z'| -- Range. Produces a machine that matches any
characters in the specified range.  Allowable upper and lower bounds of the
range are concatenation literals of length one and numerical literals.  For
example, \verb|0x10..0x20|, \verb|0..63|, and \verb|'a'..'z'| are valid ranges.
The bounds should be in the range allowed by the alphabet type.

% GENERATE: bmrange
% OPT: -p
% %%{
% machine bmrange;
\begin{comment}
\begin{verbatim}
main := 'a' .. 'z';
\end{verbatim}
\end{comment}
% }%%
% END GENERATE

\begin{center}
\includegraphics[scale=0.55]{bmrange}
\end{center}


\item \verb|variable_name| -- Lookup the machine definition assigned to the
variable name given and use an instance of it. See Section \ref{definition} for
an important note on what it means to reference a variable name.

\item \verb|builtin_machine| -- There are several built-in machines available
for use. They are all two state machines for the purpose of matching common
classes of characters. They are:

\begin{itemize}

\item \verb|any   | -- Any character in the alphabet.

\item \verb|ascii | -- Ascii characters. \verb|0..127|

\item \verb|extend| -- Ascii extended characters. This is the range
\verb|-128..127| for signed alphabets and the range \verb|0..255| for unsigned
alphabets.

\item \verb|alpha | -- Alphabetic characters. \verb|[A-Za-z]|

\item \verb|digit | -- Digits. \verb|[0-9]|

\item \verb|alnum | -- Alpha numerics. \verb|[0-9A-Za-z]|

\item \verb|lower | -- Lowercase characters. \verb|[a-z]|

\item \verb|upper | -- Uppercase characters. \verb|[A-Z]|

\item \verb|xdigit| -- Hexadecimal digits. \verb|[0-9A-Fa-f]|

\item \verb|cntrl | -- Control characters. \verb|0..31|

\item \verb|graph | -- Graphical characters. \verb|[!-~]|

\item \verb|print | -- Printable characters. \verb|[ -~]|

\item \verb|punct | -- Punctuation. Graphical characters that are not alphanumerics.
\verb|[!-/:-@[-`{-~]|

\item \verb|space | -- Whitespace. \verb|[\t\v\f\n\r ]|

\item \verb|zlen  | -- Zero length string. \verb|""|

\item \verb|empty | -- Empty set. Matches nothing. \verb|^any|

\end{itemize}
\end{itemize}

\section{Operator Precedence}
The following table shows operator precedence from lowest to highest. Operators
in the same precedence group are evaluated from left to right.

\verbspace
\begin{tabular}{|c|c|c|}
\hline
1&\verb| , |&Join\\
\hline
2&\verb/ | & - --/&Union, Intersection and Subtraction\\
\hline
3&\verb| . <: :> :>> |&Concatenation\\
\hline
4&\verb| : |&Label\\
\hline
5&\verb| -> |&Epsilon Transition\\
\hline
&\verb| >  @  $  % |&Transitions Actions and Priorities\\
\cline{2-3}
&\verb| >/  $/  %/  </  @/  <>/ |&EOF Actions\\
\cline{2-3}
6&\verb| >!  $!  %!  <!  @!  <>! |&Global Error Actions\\
\cline{2-3}
&\verb| >^  $^  %^  <^  @^  <>^ |&Local Error Actions\\
\cline{2-3}
&\verb| >~  $~  %~  <~  @~  <>~ |&To-State Actions\\
\cline{2-3}
&\verb| >*  $*  %*  <*  @*  <>* |&From-State Action\\
\hline
7&\verb| * ** ? + {n} {,n} {n,} {n,m} |&Repetition\\
\hline
8&\verb| ! ^ |&Negation and Character-Level Negation\\
\hline
9&\verb| ( <expr> ) |&Grouping\\
\hline
\end{tabular}

\section{Regular Language Operators}
\label{machconst}

When using Ragel it is helpful to have a sense of how it constructs machines.
The determinization process can produce results that seem unusual to someone
not familiar with the NFA to DFA conversion algorithm. In this section we
describe Ragel's state machine operators. Though the operators are defined
using epsilon transitions, it should be noted that this is for discussion only.
The epsilon transitions described in this section do not persist, but are
immediately removed by the determinization process which is executed in every
operation. Ragel does not make use of any nondeterministic intermediate state
machines. 

To create an epsilon transition between two states \verb|x| and \verb|y| is to
copy all of the properties of \verb|y| into \verb|x|. This involves drawing in
all of \verb|y|'s to-state actions, EOF actions, etc., in addition to its
transitions. If \verb|x| and \verb|y| both have a transition out on the same
character, then the transitions must be combined.  During transition
combination a new transition is made that goes to a new state that is the
combination of both target states. The new combination state is created using
the same epsilon transition method.  The new state has an epsilon transition
drawn to all the states that compose it. Since every time an epsilon transition
is drawn the creation of new epsilon transitions may be triggered, the process
of drawing epsilon transitions is repeated until there are no more epsilon
transitions to be made.

A very common error that is made when using Ragel is to make machines that do
too much at once. That is, to create machines that have unintentional
nondeterminism. This usually results from being unaware of the common strings
between machines that are combined together using the regular language
operators. This can involve never leaving a machine, causing its actions to be
propagated through all the following states. Or it can involve an alternation
where both branches are unintentionally taken simultaneously.

This problem forces one to think hard about the language that needs to be
matched. To guard against this kind of problem one must ensure that the machine
specification is divided up using boundaries that do not allow ambiguities from
one portion of the machine to the next. See Chapter
\ref{controlling-nondeterminism} for more on this problem and how to solve it.

The Graphviz tool is an immense help when debugging improperly compiled
machines or otherwise learning how to use Ragel. In many cases, practical
parsing programs will be too large to completely visualize with Graphviz.  The
proper approach is to reduce the language to the smallest subset possible that
still exhibits the characteristics that one wishes to learn about or to fix.
This can be done without modifying the source code using the \verb|-M| and
\verb|-S| options at the frontend. If a machine cannot be easily reduced,
embeddings of unique actions can be very useful for tracing a
particular component of a larger machine specification, since action names are
written out on transition labels.

\subsection{Union}

\verb/expr | expr/
\verbspace

The union operation produces a machine that matches any string in machine one
or machine two. The operation first creates a new start state. Epsilon
transitions are drawn from the new start state to the start states of both
input machines.  The resulting machine has a final state set equivalent to the
union of the final state sets of both input machines. In this operation, there
is the opportunity for nondeterminism among both branches. If there are
strings, or prefixes of strings that are matched by both machines then the new
machine will follow both parts of the alternation at once. The union operation is
shown below.

\graphspace
\begin{center}
\includegraphics{opor}
\end{center}
\graphspace

The following example demonstrates the union of three machines representing
common tokens.

% GENERATE: exor
% OPT: -p
% %%{
% machine exor;
\begin{inline_code}
\begin{verbatim}
# Hex digits, decimal digits, or identifiers
main := '0x' xdigit+ | digit+ | alpha alnum*;
\end{verbatim}
\end{inline_code}
% }%%
% END GENERATE

\graphspace
\begin{center}
\includegraphics[scale=0.55]{exor}
\end{center}

\subsection{Intersection}

\verb|expr & expr|
\verbspace

Intersection produces a machine that matches any
string that is in both machine one and machine two. To achieve intersection, a
union is performed on the two machines. After the result has been made
deterministic, any final state that is not a combination of final states from
both machines has its final state status revoked. To complete the operation,
paths that do not lead to a final state are pruned from the machine. Therefore,
if there are any such paths in either of the expressions they will be removed
by the intersection operator.  Intersection can be used to require that two
independent patterns be simultaneously satisfied as in the following example.

% GENERATE: exinter
% OPT: -p
% %%{
% machine exinter;
\begin{inline_code}
\begin{verbatim}
# Match lines four characters wide that contain 
# words separated by whitespace.
main :=
    /[^\n][^\n][^\n][^\n]\n/* &
    (/[a-z][a-z]*/ | [ \n])**;
\end{verbatim}
\end{inline_code}
% }%%
% END GENERATE

\graphspace
\begin{center}
\includegraphics[scale=0.55]{exinter}
\end{center}

\subsection{Difference}

\verb|expr - expr|
\verbspace

The difference operation produces a machine that matches
strings that are in machine one but are not in machine two. To achieve subtraction,
a union is performed on the two machines. After the result has been made
deterministic, any final state that came from machine two or is a combination
of states involving a final state from machine two has its final state status
revoked. As with intersection, the operation is completed by pruning any path
that does not lead to a final state.  The following example demonstrates the
use of subtraction to exclude specific cases from a set.

\verbspace

% GENERATE: exsubtr
% OPT: -p
% %%{
% machine exsubtr;
\begin{inline_code}
\begin{verbatim}
# Subtract keywords from identifiers.
main := /[a-z][a-z]*/ - ( 'for' | 'int' );
\end{verbatim}
\end{inline_code}
% }%%
% END GENERATE

\graphspace
\begin{center}
\includegraphics[scale=0.55]{exsubtr}
\end{center}
\graphspace


\subsection{Strong Difference}
\label{strong_difference}

\verb|expr -- expr|
\verbspace

Strong difference produces a machine that matches any string of the first
machine that does not have any string of the second machine as a substring. In
the following example, strong subtraction is used to excluded \verb|CRLF| from
a sequence. In the corresponding visualization, the label \verb|DEF| is short
for default. The default transition is taken if no other transition can be
taken.

% GENERATE: exstrongsubtr
% OPT: -p
% %%{
% machine exstrongsubtr;
\begin{inline_code}
\begin{verbatim}
crlf = '\r\n';
main := [a-z]+ ':' ( any* -- crlf ) crlf;
\end{verbatim}
\end{inline_code}
% }%%
% END GENERATE

\graphspace
\begin{center}
\includegraphics[scale=0.55]{exstrongsubtr}
\end{center}
\graphspace

This operator is equivalent to the following.

\verbspace
\begin{verbatim}
expr - ( any* expr any* )
\end{verbatim}

\subsection{Concatenation}

\verb|expr . expr|
\verbspace

Concatenation produces a machine that matches all the strings in machine one followed by all
the strings in machine two.  Concatenation draws epsilon transitions from the
final states of the first machine to the start state of the second machine. The
final states of the first machine lose their final state status, unless the
start state of the second machine is final as well. 
Concatenation is the default operator. Two machines next to each other with no
operator between them results in the machines being concatenated together.  

\graphspace
\begin{center}
\includegraphics{opconcat}
\end{center}
\graphspace

The opportunity for nondeterministic behaviour results from the possibility of
the final states of the first machine accepting a string that is also accepted
by the start state of the second machine.
The most common scenario that this happens in is the
concatenation of a machine that repeats some pattern with a machine that gives
a termination string, but the repetition machine does not exclude the
termination string. The example in Section \ref{strong_difference}
guards against this. Another example is the expression \verb|("'" any* "'")|.
When executed the thread of control will
never leave the \verb|any*| machine.  This is a problem especially if actions
are embedded to process the characters of the \verb|any*| component.

In the following example, the first machine is always active due to the
nondeterministic nature of concatenation. This particular nondeterminism is intended
however because we wish to permit EOF strings before the end of the input.

% GENERATE: exconcat
% OPT: -p
% %%{
% machine exconcat;
\begin{inline_code}
\begin{verbatim}
# Require an eof marker on the last line.
main := /[^\n]*\n/* . 'EOF\n';
\end{verbatim}
\end{inline_code}
% }%%
% END GENERATE

\graphspace
\begin{center}
\includegraphics[scale=0.55]{exconcat}
\end{center}
\graphspace

\noindent {\bf Note:} There is a language
ambiguity involving concatenation and subtraction. Because concatenation is the 
default operator for two
adjacent machines there is an ambiguity between subtraction of
a positive numerical literal and concatenation of a negative numerical literal.
For example, \verb|(x-7)| could be interpreted as \verb|(x . -7)| or 
\verb|(x - 7)|. In the Ragel language, the subtraction operator always takes precedence
over concatenation of a negative literal. Precedence was given to the
subtraction-based interpretation so as to adhere to the rule that the default
concatenation operator takes effect only when there are no other operators between
two machines. Beware of writing machines such as \verb|(any -1)| when what is
desired is a concatenation of \verb|any| and -1. Instead write 
\verb|(any .  -1)| or \verb|(any (-1))|. If in doubt of the meaning of your program do not
rely on the default concatenation operator; always use the \verb|.| symbol.


\subsection{Kleene Star}

\verb|expr*|
\verbspace

The machine resulting from the Kleene Star operator will match zero or more
repetitions of the machine it is applied to.
It creates a new start state and an additional final
state.  Epsilon transitions are drawn between the new start state and the old start
state, between the new start state and the new final state, and
between the final states of the machine and the new start state.  After the
machine is made deterministic the effect is of the final states getting all the
transitions of the start state. 

\graphspace
\begin{center}
\includegraphics{opstar}
\end{center}
\graphspace

The possibility for nondeterministic behaviour arises if the final states have
transitions on any of the same characters as the start state.  This is common
when applying kleene star to an alternation of tokens. Like the other problems
arising from nondeterministic behavior, this is discussed in more detail in Chapter
\ref{controlling-nondeterminism}. This particular problem can also be solved
by using the longest-match construction discussed in Section 
\ref{generating-scanners} on scanners.

In this simple
example, there is no nondeterminism introduced by the exterior kleene star due to
the newline at the end of the regular expression. Without the newline the
exterior kleene star would be redundant and there would be ambiguity between
repeating the inner range of the regular expression and the entire regular
expression. Though it would not cause a problem in this case, unnecessary
nondeterminism in the kleene star operator often causes undesired results for
new Ragel users and must be guarded against.

% GENERATE: exstar
% OPT: -p
% %%{
% machine exstar;
\begin{inline_code}
\begin{verbatim}
# Match any number of lines with only lowercase letters.
main := /[a-z]*\n/*;
\end{verbatim}
\end{inline_code}
% }%%
% END GENERATE

\graphspace
\begin{center}
\includegraphics[scale=0.55]{exstar}
\end{center}
\graphspace

\subsection{One Or More Repetition}

\verb|expr+|
\verbspace

This operator produces the concatenation of the machine with the kleene star of
itself. The result will match one or more repetitions of the machine. The plus
operator is equivalent to \verb|(expr . expr*)|.  The plus operator makes
repetitions that cannot be zero length.

% GENERATE: explus
% OPT: -p
% %%{
% machine explus;
\begin{inline_code}
\begin{verbatim}
# Match alpha-numeric words.
main := alnum+;
\end{verbatim}
\end{inline_code}
% }%%
% END GENERATE

\graphspace
\begin{center}
\includegraphics[scale=0.55]{explus}
\end{center}
\graphspace

\subsection{Optional}

\verb|expr?|
\verbspace

The {\em optional} operator produces a machine that accepts the machine
given or the zero length string. The optional operator is equivalent to
\verb/(expr | '' )/. In the following example the optional operator is used to
extend a token.

% GENERATE: exoption
% OPT: -p
% %%{
% machine exoption;
\begin{inline_code}
\begin{verbatim}
# Match integers or floats.
main := digit+ ('.' digit+)?;
\end{verbatim}
\end{inline_code}
% }%%
% END GENERATE

\graphspace
\begin{center}
\includegraphics[scale=0.55]{exoption}
\end{center}
\graphspace


\subsection{Repetition}

\begin{tabbing}
\noindent \verb|expr {n}| \hspace{16pt}\=-- Exactly N copies of expr.\\

\noindent \verb|expr {,n}| \>-- Zero to N copies of expr.\\

\noindent \verb|expr {n,}| \>-- N or more copies of expr.\\

\noindent \verb|expr {n,m}| \>-- N to M copies of expr.
\end{tabbing}

\subsection{Negation}

\verb|!expr|
\verbspace

Negation produces a machine that matches any string not matched by the given
machine. Negation is equivalent to \verb|(any* - expr)|.

% GENERATE: exnegate
% OPT: -p
% %%{
% machine exnegate;
\begin{inline_code}
\begin{verbatim}
# Accept anything but a string beginning with a digit.
main := ! ( digit any* );
\end{verbatim}
\end{inline_code}
% }%%
% END GENERATE

\graphspace
\begin{center}
\includegraphics[scale=0.55]{exnegate}
\end{center}
\graphspace


\subsection{Character-Level Negation}

\verb|^expr|
\verbspace

Character-level negation produces a machine that matches any single character
not matched by the given machine. Character-Level Negation is equivalent to
\verb|(any - expr)|.

\section{State Machine Minimization}

State machine minimization is the process of finding the minimal equivalent FSM accepting
the language. Minimization reduces the number of states in machines
by merging equivalent states. It does not change the behaviour of the machine
in any way. It will cause some states to be merged into one because they are
functionally equivalent. State minimization is on by default. It can be turned
off with the \verb|-n| option.

The algorithm implemented is similar to Hopcroft's state minimization
algorithm. Hopcroft's algorithm assumes a finite alphabet that can be listed in
memory, whereas Ragel supports arbitrary integer alphabets that cannot be
listed in memory. Though exact analysis is very difficult, Ragel minimization
runs close to $O(n \times log(n))$ and requires $O(n)$ temporary storage where
$n$ is the number of states.

\section{Visualization}

Ragel is able to emit compiled state machines in Graphviz's Dot file format.
Graphviz support allows users to perform
incremental visualization of their parsers. User actions are displayed on
transition labels of the graph. If the final graph is too large to be
meaningful, or even drawn, the user is able to inspect portions of the parser
by naming particular regular expression definitions with the \verb|-S| and
\verb|-M| options to the \verb|ragel| program. Use of Graphviz greatly
improves the Ragel programming experience. It allows users to learn Ragel by
experimentation and also to track down bugs caused by unintended
nondeterminism.

\chapter{User Actions}

Ragel permits the user to embed actions into the transitions of a regular
expression's corresponding state machine. These actions are executed when the
generated code moves over a transition.  Like the regular expression operators,
the action embedding operators are fully compositional. They take a state
machine and an action as input, embed the action, and yield a new state machine
that can be used in the construction of other machines. Due to the
compositional nature of embeddings, the user has complete freedom in the
placement of actions.

A machine's transitions are categorized into four classes, The action embedding
operators access the transitions defined by these classes.  The {\em starting
transition} operator \verb|>| isolates the start state, then embeds an action
into all transitions leaving it. The {\em finishing transition} operator
\verb|@| embeds an action into all transitions going into a final state.  The
{\em all transition} operator \verb|$| embeds an action into all transitions of
an expression. The {\em pending out transition} operator \verb|%| provides
access to yet-unmade leaving transitions. 

\section{Embedding Actions}

\begin{verbatim}
action ActionName {
    /* Code an action here. */
    count += 1;
}
\end{verbatim}
\verbspace

The action statement defines a block of code that can be embedded into an FSM.
Action names can be referenced by the action embedding operators in
expressions. Though actions need not be named in this way (literal blocks
of code can be embedded directly when building machines), defining reusable
blocks of code whenever possible is good practice because it potentially increases the
degree to which the machine can be minimized. Within an action some Ragel expressions
and statements are parsed and translated. These allow the user to interact with the machine
from action code. See Section \ref{vals} for a complete list of statements and
values available in code blocks. 

\subsection{Starting Action}

\verb|expr > action| 
\verbspace

The starting transition operator embeds an action into all transitions that
leave the start state. In some machines the start state has in transtions from
within the machine and the start state is effectively reused. In these cases
the start state is first isolated from the rest of the machine and the starting
actions do not get re-executed.

If the start state is a final state then it is possible for the machine to
never be started and the starting transitions by-passed.  In the following
example, the action is executed on the first transition of the machine. If the
repetition machine is bypassed the action is not executed.

\verbspace

% GENERATE: exstact
% OPT: -p
% %%{
% machine exstact;
\begin{inline_code}
\begin{verbatim}
# Execute A at the beginning of a string of alpha.
action A {}
main := ( lower* >A ) . ' ';
\end{verbatim}
\end{inline_code}
% }%%
% END GENERATE

\graphspace
\begin{center}
\includegraphics[scale=0.55]{exstact}
\end{center}
\graphspace

\subsection{Finishing Action}

\verb|expr @ action|
\verbspace

The finishing action operator embeds an action into any transitions that go into a
final state. Whether or not the machine accepts is not determined at the point
the action is executed. Further input may move the machine out of the accepting
state, but keep it in the machine. As in the following example, the
into-final-state operator is most often used when no lookahead is necessary.

% GENERATE: exdoneact
% OPT: -p
% %%{
% machine exdoneact;
% action A {}
\begin{inline_code}
\begin{verbatim}
# Execute A when the trailing space is seen.
main := ( lower* ' ' ) @A;
\end{verbatim}
\end{inline_code}
% }%%
% END GENERATE

\graphspace
\begin{center}
\includegraphics[scale=0.55]{exdoneact}
\end{center}
\graphspace


\subsection{All Transition Action}

\verb|expr $ action|
\verbspace

The all transition operator embeds an action into all transitions of a machine.
The action is executed whenever a transition of the machine is taken. In the
following example, A is executed on every character matched.

% GENERATE: exallact
% OPT: -p
% %%{
% machine exallact;
% action A {}
\begin{inline_code}
\begin{verbatim}
# Execute A on any characters of machine one or two.
main := ( 'm1' | 'm2' ) $A;
\end{verbatim}
\end{inline_code}
% }%%
% END GENERATE

\graphspace
\begin{center}
\includegraphics[scale=0.55]{exallact}
\end{center}
\graphspace


\subsection{Pending Out (Leaving) Actions}
\label{out-actions}

\verb|expr % action|
\verbspace

The pending out action operator embeds an action into the pending out
transitions of a machine. The action is first embedded into the final states of
the machine and later transferred to any transitions made going out of the
machine. The transfer can be caused either by a concatenation or kleene star
operation.  This mechanism allows one to associate an action with the
termination of a sequence, without being concerned about what particular
character terminates the sequence.  In the following example, A is executed
when leaving the alpha machine by the newline character.

% GENERATE: exoutact1
% OPT: -p
% %%{
% machine exoutact1;
% action A {}
\begin{inline_code}
\begin{verbatim}
# Match a word followed by a newline. Execute A when 
# finishing the word.
main := ( lower+ %A ) . '\n';
\end{verbatim}
\end{inline_code}
% }%%
% END GENERATE

\graphspace
\begin{center}
\includegraphics[scale=0.55]{exoutact1}
\end{center}
\graphspace

In the following example, the \verb|term_word| action could be used to register
the appearance of a word and to clear the buffer that the \verb|lower| action used
to store the text of it.

% GENERATE: exoutact2
% OPT: -p
% %%{
% machine exoutact2;
% action lower {}
% action space {}
% action term_word {}
% action newline {}
\begin{inline_code}
\begin{verbatim}
word = ( [a-z] @lower )+ %term_word;
main := word ( ' ' @space word )* '\n' @newline;
\end{verbatim}
\end{inline_code}
% }%%
% END GENERATE

\graphspace
\begin{center}
\includegraphics[scale=0.55]{exoutact2}
\end{center}
\graphspace


In this final example of the action embedding operators, A is executed upon the
first character of the alpha machine, B is executed on all transitions of the
alpha machine, C is executed when the alpha machine is exited by moving into the
newline machine and N is executed when the newline machine moves into a final
state.  

% GENERATE: exaction
% OPT: -p
% %%{
% machine exaction;
% action A {}
% action B {}
% action C {}
% action N {}
\begin{inline_code}
\begin{verbatim}
# Execute A on starting the alpha machine, B on every transition 
# moving through it and C upon finishing. Execute N on the newline.
main := ( lower* >A $B %C ) . '\n' @N;
\end{verbatim}
\end{inline_code}
% }%%
% END GENERATE

\graphspace
\begin{center}
\includegraphics[scale=0.55]{exaction}
\end{center}
\graphspace


\section{State Action Embedding Operators}

The state embedding operators allow one to embed actions into states. Like the
transition embedding operators, there are several different classes of states
that the operators access. The meanings of the symbols are similar to the
meanings of the symbols used by the transition embedding operators. The design
of the state selections was driven by a need to cover the states of an
expression with a single error action.

Unlike the transition embedding operators, the state embedding operators are
also distinguished by the different kinds of events that embedded actions can
be associated with. Therefore the state embedding operators have two
components.  The first, which is the first one or two characters, specifies the
class of states that the action will be embedded into. The second component
specifies the type of event the action will be executed on. The symbols of the
second component also have equivalent kewords. 

\def\fakeitem{\hspace*{12pt}$\bullet$\hspace*{10pt}}

\begin{minipage}{\textwidth}
\begin{multicols}{2}
\raggedcolumns
\noindent The different classes of states are:\\
\fakeitem \verb|> | -- the start state \\
\fakeitem \verb|$ | -- all states\\
\fakeitem \verb|% | -- final states\\
\fakeitem \verb|< | -- any state except the start state\\
\fakeitem \verb|@ | -- any state except final states\\
\fakeitem \verb|<>| -- any except start and final (middle)

\columnbreak

\noindent The different kinds of embeddings are:\\
\fakeitem \verb|~| -- to-state actions (\verb|to|)\\
\fakeitem \verb|*| -- from-state actions (\verb|from|)\\
\fakeitem \verb|/| -- EOF actions (\verb|eof|)\\
\fakeitem \verb|!| -- error actions (\verb|err|)\\
\fakeitem \verb|^| -- local error actions (\verb|lerr|)\\
\end{multicols}
\end{minipage}
%\label{state-act-embed}
%\caption{The two components of state embedding operators. The class of states
%to select comes first, followed by the type of embedding.}
%
%\begin{figure}[t]
%\centering
%\includegraphics{stembed}
%\caption{Summary of state manipulation operators}
%\label{state-act-embed-chart}
%\end{figure}

%\noindent Putting these two components together we get a matrix of state
%embedding operators. The entire set is given in Figure \ref{state-act-embed-chart}.


\subsection{To-State and From-State Actions}

\subsubsection{To-State Actions}

\noindent\verb|>~action     <~action     $~action    %~action      @~action      <>~action|\\
\\
\noindent Verbose forms:\\
\noindent\verb|>to(act)     <to(act)     $to(na)     %to(name)     @to(name)     <>to(name)|\\
\noindent\verb|>to{...}     <to{...}     $to{...}    %to{...}      @to{...}      <>to{...}|
\\


To-state actions are executed whenever the state machine moves into the
specified state, either by a natural movement over a transition or by an
action-based transfer of control such as \verb|fgoto|. They are executed after the
in-transition's actions but before the current character is advanced and
tested against the end of the input block. To-state embeddings stay with the
state. They are irrespective of the state's current set of transitions and any
future transitions that may be added in or out of the state.

Note that the setting of the current state variable \verb|cs| outside of the
execute code is not considered by Ragel as moving into a state and consequently
the to-state actions of the new current state are not executed. This includes
the initialization of the current state when the machine begins.  This is
because the entry point into the machine execution code is after the execution
of to-state actions.

\subsubsection{From-State Actions}

\noindent\verb|>*action     <*action     $*action    %*action      @*action      <>*action|\\
\\
\noindent Verbose forms:\\
\noindent\verb|>from(act)   <from(act)   $from(na)   %from(name)   @from(name)   <>from(name)|\\
\noindent\verb|>from{...}   <from{...}   $from{...}  %from{...}    @from{...}    <>from{...}|
\\

From-state actions are executed whenever the state machine takes a transition from a
state, either to itself or to some other state. These actions are executed
immediately after the current character is tested against the input block end
marker and before the transition to take is sought based on the current
character. From-state actions are therefore executed even if a transition
cannot be found and the machine moves into the error state.  Like to-state
embeddings, from-state embeddings stay with the state.

\subsection{EOF Actions}

\noindent\verb|>/action     </action     $/action    %/action      @/action      <>/action|\\
\\
\noindent Verbose forms:\\
\noindent\verb|>eof(act)    <eof(act)    $eof(na)    %eof(name)    @eof(name)    <>eof(name)|\\
\noindent\verb|>eof{...}    <eof{...}    $eof{...}   %eof{...}     @eof{...}     <>eof{...}|
\\


The EOF action embedding operators enable the user to embed EOF actions into
different classes of
states.  EOF actions are stored in states and generated with the \verb|write eof|
statement. The generated EOF code switches on the current state and executes the EOF
actions associated with it.

\subsection{Handling Errors}

In many applications it is useful to be able to react to parsing errors.  The
user may wish to print an error message that depends on the context.  It
may also be desirable to consume input in an attempt to return the input stream
to some known state and resume parsing. To support error handling and recovery,
Ragel provides error action embedding operators. There are two kinds of error
actions, regular (global) error actions and local error actions.
Error actions can be used to simply report errors, or by jumping to a machine
instantiation that consumes input, can attempt to recover from errors.  

\subsubsection{Global Error Actions}

\noindent\verb|>!action     <!action     $!action    %!action      @!action      <>!action|\\
\\
\noindent Verbose forms:\\
\noindent\verb|>err(act)    <err(act)    $err(na)    %err(name)    @err(name)    <>err(name)|\\
\noindent\verb|>err{...}    <err{...}    $err{...}   %err{...}     @err{...}     <>err{...}|
\\

Error actions are stored in states until the final state machine has been fully
constructed. They are then transferred to the transitions that move into the
error state. This transfer entails the creation of a transition from the state
to the error state that is taken on all input characters that are not already
covered by the state's transitions. In other words it provides a default
action. Error actions can induce a recovery by altering \verb|p| and then jumping back
into the machine with \verb|fgoto|.

\subsubsection{Local Error Actions}

\noindent\verb|>^action     <^action     $^action    %^action      @^action      <>^action|\\
\\
\noindent Verbose forms:\\
\noindent\verb|>lerr(act)   <lerr(act)   $lerr(na)   %lerr(name)   @lerr(name)   <>lerr(name)|\\
\noindent\verb|>lerr{...}   <lerr{...}   $lerr{...}  %lerr{...}    @lerr{...}    <>lerr{...}|
\\

Like global error actions, local error actions are also stored in states until
a transfer point. The transfer point is different however. Each local error action
embedding is associated with a name. When a machine definition has been fully
constructed, all local error action embeddings associated the same name as the
machine are transferred to error transitions. Local error actions can be used
to specify an action to take when a particular section of a larger state
machine fails to make a match. A particular machine definition's ``thread'' may
die and the local error actions executed, however the machine as a whole may
continue to match input.

There are two forms of local error action embeddings. In the first form the name defaults
to the current machine. In the second form the machine name can be specified.  This
is useful when it is more convenient to specify the local error action in a
sub-definition that is used to construct the machine definition where the
transfer should happen. To embed local error actions and explicitly state the
machine on which the transfer is to happen use \verb|(name, action)| as the
action.

\begin{comment}
\begin{itemize}
\setlength{\parskip}{0in}
\item \verb|expr >^ (name, action) | -- Start state.
\item \verb|expr $^ (name, action) | -- All states.
\item \verb|expr %^ (name, action) | -- Final states.
\item \verb|expr <^ (name, action) | -- Not start state.
\item \verb|expr <>^ (name, action)| -- Not start and not final states.
\end{itemize}
\end{comment}

\subsubsection{Example}

The following example uses error actions to report an error and jump to a
machine that consumes the remainder of the line when parsing fails. After
consuming the line, the error recovery machine returns to the main loop.

% GENERATE: erract
% %%{
% 	machine erract;
%	ws = ' ';
%	address = 'foo@bar.com';
%	date = 'Monday May 12';
\begin{inline_code}
\begin{verbatim}
action cmd_err { 
    printf( "command error\n" ); 
    fhold; fgoto line;
}
action from_err { 
    printf( "from error\n" ); 
    fhold; fgoto line; 
}
action to_err { 
    printf( "to error\n" ); 
    fhold; fgoto line;
}

line := [^\n]* '\n' @{ fgoto main; };

main := (
    (
        'from' @err(cmd_err) 
            ( ws+ address ws+ date '\n' ) $err(from_err) |
        'to' @err(cmd_err)
            ( ws+ address '\n' ) $err(to_err)
    ) 
)*;
\end{verbatim}
\end{inline_code}
% }%%
% %% write data;
% void f()
% {
% 	%% write init;
% 	%% write exec;
% }
% END GENERATE



\section{Action Ordering and Duplicates}

When building a parser by combining smaller expressions that themselves have
embedded actions, it is often the case that transitions that need to
execute a number of actions on one input character are made. For example when we leave
an expression, we may execute the expression's pending out action and the
subsequent expression's starting action on the same input character.  We must
therefore devise a method for ordering actions that is both intuitive and
predictable for the user and repeatable by the state machine compiler. The
determinization processes cannot simply order actions by the time at which they
are introduced into a transition -- otherwise the programmer will be at the
mercy of luck.

We associate with the embedding of each action a distinct timestamp that is
used to order actions that appear together on a single transition in the final
compiled state machine. To accomplish this we traverse the parse tree of
regular expressions and assign timestamps to action embeddings. This algorithm
is recursive in nature and quite simple. When it visits a parse tree node it
assigns timestamps to all {\em starting} action embeddings, recurses on the
parse tree, then assigns timestamps to the remaining {\em all}, {\em
finishing}, and {\em leaving} embeddings in the order in which they appear.

Ragel does not permit actions (defined or unnamed) to appear multiple times in
an action list.  When the final machine has been created, actions that appear
more than once in a single transition or EOF action list have their duplicates
removed. The first appearance of the action is preserved. This is useful in a
number of scenarios.  First, it allows us to union machines with common
prefixes without worrying about the action embeddings in the prefix being
duplicated.  Second, it prevents pending out actions from being transferred multiple times
when a concatenation follows a kleene star and the two machines begin with a common
character.

\verbspace
\begin{verbatim}
word = [a-z]+ %act;
main := word ( '\n' word )* '\n\n';
\end{verbatim}

\section{Values and Statements Available in Code Blocks}
\label{vals}

\noindent The following values are available in code blocks:

\begin{itemize}
\item \verb|fpc| -- A pointer to the current character. This is equivalent to
accessing the \verb|p| variable.

\item \verb|fc| -- The current character. This is equivalent to the expression \verb|(*p)|.

\item \verb|fcurs| -- An integer value representing the current state. This
value should only be read from. To move to a different place in the machine
from action code use the \verb|fgoto|, \verb|fnext| or \verb|fcall| statements.
Outside of the machine execution code the \verb|cs| variable may be modified.

\item \verb|ftargs| -- An integer value representing the target state. This
value should only be read from. Again, \verb|fgoto|, \verb|fnext| and
\verb|fcall| can be used to move to a specific entry point.

\item \verb|fentry(<label>)| -- Retrieve an integer value representing the
entry point \verb|label|. The integer value returned will be a compile time
constant. This number is suitable for later use in control flow transfer
statements that take an expression. This value should not be compared against
the current state because any given label can have multiple states representing
it. The value returned by \verb|fentry| will be one of the possibly multiple states the
label represents.
\end{itemize}

\noindent The following statements are available in code blocks:

\begin{itemize}

\item \verb|fhold;| -- Do not advance over the current character. If processing
data in multiple buffer blocks, the \verb|fhold| statement should only be used
once in the set of actions executed on a character.  Multiple calls may result
in backing up over the beginning of the buffer block. The \verb|fhold|
statement does not imply any transfer of control. It is equivalent to the
\verb|p--;| statement. 

\item \verb|fexec <expr>;| -- Set the next character to process. This can be
used to backtrack to previous input or advance ahead.
Unlike \verb|fhold|, which can be used
anywhere, \verb|fexec| requires the user to ensure that the target of the
backtrack is in the current buffer block or is known to be somewhere ahead of
it. The machine will continue iterating forward until \verb|pe| is arrived at,
\verb|fbreak| is called or the machine moves into the error state. In actions
embedded into transitions, the \verb|fexec| statement is equivalent to setting
\verb|p| to one position ahead of the next character to process.  If the user
also modifies \verb|pe|, it is possible to change the buffer block entirely.

\item \verb|fgoto <label>;| -- Jump to an entry point defined by
\verb|<label>|.  The \verb|fgoto| statement immediately transfers control to
the destination state.

\item \verb|fgoto *<expr>;| -- Jump to an entry point given by \verb|<expr>|.
The expression must evaluate to an integer value representing a state.

\item \verb|fnext <label>;| -- Set the next state to be the entry point defined
by \verb|label|.  The \verb|fnext| statement does not immediately jump to the
specified state. Any action code following the statement is executed.

\item \verb|fnext *<expr>;| -- Set the next state to be the entry point given
by \verb|<expr>|. The expression must evaluate to an integer value representing
a state.

\item \verb|fcall <label>;| -- Push the target state and jump to the entry
point defined by \verb|<label>|.  The next \verb|fret| will jump to the target
of the transition on which the call was made. Use of \verb|fcall| requires
the declaration of a call stack. An array of integers named \verb|stack| and a
single integer named \verb|top| must be declared. With the \verb|fcall|
construct, control is immediately transferred to the destination state.

\item \verb|fcall *<expr>;| -- Push the current state and jump to the entry
point given by \verb|<expr>|. The expression must evaluate to an integer value
representing a state.

\item \verb|fret;| -- Return to the target state of the transition on which the
last \verb|fcall| was made.  Use of \verb|fret| requires the declaration of a
call stack with \verb|fstack| in the struct block.  Control is immediately
transferred to the destination state.

\item \verb|fbreak;| -- Save the current state and immediately break out of the
execute loop. This statement is useful in conjunction with the \verb|noend|
write option. Rather than process input until the end marker of the input
buffer is arrived at, the fbreak statement can be used to stop processing input
upon seeing some end-of-string marker.  It can also be used for handling
exceptional circumstances.  The fbreak statement does not change the pointer to
the current character. After an \verb|fbreak| call the \verb|p| variable will point to
the character that was being traversed over when the action was
executed. The current state will be the target of the current transition.

\end{itemize}

\noindent {\bf Note:} Once actions with control-flow commands are embedded into a
machine, the user must exercise caution when using the machine as the operand
to other machine construction operators. If an action jumps to another state
then unioning any transition that executes that action with another transition
that follows some other path will cause that other path to be lost. Using
commands that manually jump around a machine takes us out of the domain of
regular languages because transitions that may be conditional and that the
machine construction operators are not aware of are introduced.  These
commands should therefore be used with caution.


\chapter{Controlling Nondeterminism}
\label{controlling-nondeterminism}

Along with the flexibility of arbitrary action embeddings comes a need to
control nondeterminism in regular expressions. If a regular expression is
ambiguous, then sub-components of a parser other than the intended parts may become
active. This means that actions that are irrelevant to the
current subset of the parser may be executed, causing problems for the
programmer.

Tools that are based on regular expression engines and used for
recognition tasks will usually function as intended regardless of the presence
of ambiguities. It is quite common for users of scripting languages to write
regular expressions that are heavily ambiguous and it generally does not
matter. As long as one of the potential matches is recognized, there can be any
number of other matches present.  In some parsing systems the run-time engine
can employ a strategy for resolving ambiguities, for example always pursuing
the longest possible match and discarding others.

In Ragel, there is no regular expression run-time engine, just a simple state
machine execution model. When we begin to embed actions and face the
possibility of spurious action execution, it becomes clear that controlling
nondeterminism at the machine construction level is very important. Consider
the following example.

% GENERATE: lines1
% OPT: -p
% %%{
% machine lines1;
% action first {}
% action tail {}
% word = [a-z]+;
\begin{inline_code}
\begin{verbatim}
ws = [\n\t ];
line = word $first ( ws word $tail )* '\n';
lines = line*;
\end{verbatim}
\end{inline_code}
% main := lines;
% }%%
% END GENERATE

\begin{center}
\includegraphics[scale=0.53]{lines1}
\end{center}
\graphspace

Since the \verb|ws| expression includes the newline character, we will
not finish the \verb|line| expression when a newline character is seen. We will
simultaneously pursue the possibility of matching further words on the same
line and the possibility of matching a second line. Evidence of this fact is 
in the state tables. On several transitions both the \verb|first| and
\verb|tail| actions are executed.  The solution here is simple: exclude
the newline character from the \verb|ws| expression. 

% GENERATE: lines2
% OPT: -p
% %%{
% machine lines2;
% action first {}
% action tail {}
% word = [a-z]+;
\begin{inline_code}
\begin{verbatim}
ws = [\t ];
line = word $first ( ws word $tail )* '\n';
lines = line*;
\end{verbatim}
\end{inline_code}
% main := lines;
% }%%
% END GENERATE

\begin{center}
\includegraphics[scale=0.55]{lines2}
\end{center}
\graphspace

Solving this kind of problem is straightforward when the ambiguity is created
by strings that are a single character long.  When the ambiguity is created by
strings that are multiple characters long we have a more difficult problem.
The following example is an incorrect attempt at a regular expression for C
language comments. 

% GENERATE: comments1
% OPT: -p
% %%{
% machine comments1;
% action comm {}
\begin{inline_code}
\begin{verbatim}
comment = '/*' ( any @comm )* '*/';
main := comment ' ';
\end{verbatim}
\end{inline_code}
% }%%
% END GENERATE

\begin{center}
\includegraphics[scale=0.55]{comments1}
\end{center}
\graphspace

Using standard concatenation, we will never leave the \verb|any*| expression.
We will forever entertain the possibility that a \verb|'*/'| string that we see
is contained in a longer comment and that, simultaneously, the comment has
ended.  The concatenation of the \verb|comment| machine with \verb|SP| is done
to show this. When we match space, we are also still matching the comment body.

One way to approach the problem is to exclude the terminating string
from the \verb|any*| expression using set difference. We must be careful to
exclude not just the terminating string, but any string that contains it as a
substring. A verbose, but proper specification of a C comment parser is given
by the following regular expression. 

% GENERATE: comments2
% OPT: -p
% %%{
% machine comments2;
% action comm {}
\begin{inline_code}
\begin{verbatim}
comment = '/*' ( ( any @comm )* - ( any* '*/' any* ) ) '*/';
\end{verbatim}
\end{inline_code}
% main := comment;
% }%%
% END GENERATE

\graphspace
\begin{center}
\includegraphics[scale=0.55]{comments2}
\end{center}
\graphspace

Note that Ragel's strong subtraction operator \verb|--| can also be used here.
In doing this subtraction we have phrased the problem of controlling non-determinism in
terms of excluding strings common to two expressions that interact when
combined.
We can also phrase the problem in terms of the transitions of the state
machines that implement these expressions. During the concatenation of
\verb|any*| and \verb|'*/'| we will be making transitions that are composed of
both the loop of the first expression and the final character of the second.
At this time we want the transition on the \verb|'/'| character to take precedence
over and disallow the transition that originated in the \verb|any*| loop.

In another parsing problem, we wish to implement a lightweight tokenizer that we can
utilize in the composition of a larger machine. For example, some HTTP headers
have a token stream as a sub-language. The following example is an attempt
at a regular expression-based tokenizer that does not function correctly due to
unintended nondeterminism.

\newpage

% GENERATE: smallscanner
% OPT: -p
% %%{
% machine smallscanner;
% action start_str {}
% action on_char {}
% action finish_str {}
\begin{inline_code}
\begin{verbatim}
header_contents = ( 
    lower+ >start_str $on_char %finish_str | 
    ' '
)*;
\end{verbatim}
\end{inline_code}
% main := header_contents;
% }%%
% END GENERATE

\begin{center}
\includegraphics[scale=0.55]{smallscanner}
\end{center}
\graphspace

In this case, the problem with using a standard kleene star operation is that
there is an ambiguity between extending a token and wrapping around the machine
to begin a new token. Using the standard operator, we get an undesirable
nondeterministic behaviour. Evidence of this can be seen on the transition out
of state one to itself.  The transition extends the string, and simultaneously,
finishes the string only to immediately begin a new one.  What is required is
for the
transitions that represent an extension of a token to take precedence over the
transitions that represent the beginning of a new token. For this problem
there is no simple solution that uses standard regular expression operators.

\section{Priorities}

A priority mechanism was devised and built into the determinization
process, specifically for the purpose of allowing the user to control
nondeterminism.  Priorities are integer values embedded into transitions. When
the determinization process is combining transitions that have different
priorities, the transition with the higher priority is preserved and the
transition with the lower priority is dropped.

Unfortunately, priorities can have unintended side effects because their
operation requires that they linger in transitions indefinitely. They must linger
because the Ragel program cannot know when the user is finished with a priority
embedding.  A solution whereby they are explicitly deleted after use is
conceivable; however this is not very user-friendly.  Priorities were therefore
made into named entities. Only priorities with the same name are allowed to
interact.  This allows any number of priorities to coexist in one machine for
the purpose of controlling various different regular expression operations and
eliminates the need to ever delete them. Such a scheme allows the user to
choose a unique name, embed two different priority values using that name
and be confident that the priority embedding will be free of any side effects.

In the first form of priority embedding the name defaults to the name of the machine
definition that the priority is assigned in. In this sense priorities are by
default local to the current machine definition or instantiation. Beware of
using this form in a longest-match machine, since there is only one name for
the entire set of longest match patterns. In the second form the priority's
name can be specified, allowing priority interaction across machine definition
boundaries.

\begin{itemize}
\setlength{\parskip}{0in}
\item \verb|expr > int| -- Sets starting transitions to have priority int.
\item \verb|expr @ int| -- Sets transitions that go into a final state to have priority int. 
\item \verb|expr $ int| -- Sets all transitions to have priority int.
\item \verb|expr % int| -- Sets pending out transitions from final states to
have priority int.\\ When a transition is made going out of the machine (either
by concatenation or kleene star) its priority is immediately set to the pending
out priority.  
\end{itemize}

The second form of priority assignment allows the programmer to specify the name
to which the priority is assigned.

\begin{itemize}
\setlength{\parskip}{0in}
\item \verb|expr > (name, int)| -- Starting transitions.
\item \verb|expr @ (name, int)| -- Finishing transitions (into a final state).
\item \verb|expr $ (name, int)| -- All transitions.
\item \verb|expr % (name, int)| -- Pending out transitions.
\end{itemize}

\section{Guarded Operators that Encapsulate Priorities}

Priority embeddings are a very expressive mechanism. At the same time they
can be very confusing for the user. They force the user to imagine
the transitions inside two interacting expressions and work out the precise
effects of the operations between them. When we consider
that this problem is worsened by the
potential for side effects caused by unintended priority name collisions, we
see that exposing the user to priorities is rather undesirable.

Fortunately, in practice the use of priorities has been necessary only in a
small number of scenarios.  This allows us to encapsulate their functionality
into a small set of operators and fully hide them from the user. This is
advantageous from a language design point of view because it greatly simplifies
the design.  

Going back to the C comment example, we can now properly specify
it using a guarded concatenation operator which we call {\em finish-guarded
concatenation}. From the user's point of view, this operator terminates the
first machine when the second machine moves into a final state.  It chooses a
unique name and uses it to embed a low priority into all
transitions of the first machine. A higher priority is then embedded into the
transitions of the second machine that enter into a final state. The following
example yields a machine identical to the example in Section 
\ref{controlling-nondeterminism}.

\begin{inline_code}
\begin{verbatim}
comment = '/*' ( any @comm )* :>> '*/';
\end{verbatim}
\end{inline_code}

\graphspace
\begin{center}
\includegraphics[scale=0.55]{comments2}
\end{center}
\graphspace

Another guarded operator is {\em left-guarded concatenation}, given by the
\verb|<:| compound symbol. This operator places a higher priority on all
transitions of the first machine. This is useful if one must forcibly separate
two lists that contain common elements. For example, one may need to tokenize a
stream, but first consume leading whitespace.

Ragel also includes a {\em longest-match kleene star} operator, given by the
\verb|**| compound symbol. This 
guarded operator embeds a high
priority into all transitions of the machine. 
A lower priority is then embedded into pending out transitions
(in a manner similar to pending out action embeddings, described in Section
\ref{out-actions}).  When the kleene star operator makes the epsilon transitions from
the final states into the start state, the lower priority will be transferred
to the epsilon transitions. In cases where following an epsilon transition
out of a final state conflicts with an existing transition out of a final
state, the epsilon transition will be dropped.

Other guarded operators are conceivable, such as guards on union that cause one
alternative to take precedence over another. These may be implemented when it
is clear they constitute a frequently used operation.
In the next section we discuss the explicit specification of state machines
using state charts.

\subsection{Entry-Guarded Concatenation}

\verb|expr :> expr| 
\verbspace

This operator concatenates two machines, but first assigns a low
priority to all transitions
of the first machine and a high priority to the starting transitions of the
second machine. This operator is useful if from the final states of the first
machine, it is possible to accept the characters in the start transitions of
the second machine. This operator effectively terminates the first machine
immediately upon starting the second machine, where otherwise they would be
pursued concurrently. In the following example, entry-guarded concatenation is
used to move out of a machine that matches everything at the first sign of an
end-of-input marker.

% GENERATE: entryguard
% OPT: -p
% %%{
% machine entryguard;
\begin{inline_code}
\begin{verbatim}
# Leave the catch-all machine on the first character of FIN.
main := any* :> 'FIN';
\end{verbatim}
\end{inline_code}
% }%%
% END GENERATE

\begin{center}
\includegraphics[scale=0.55]{entryguard}
\end{center}
\graphspace

Entry-guarded concatenation is equivalent to the following:

\verbspace
\begin{verbatim}
expr $(unique_name,0) . expr >(unique_name,1)
\end{verbatim}

\subsection{Finish-Guarded Concatenation}

\verb|expr :>> expr|
\verbspace

This operator is
like the previous operator, except the higher priority is placed on the final
transitions of the second machine. This is useful if one wishes to entertain
the possibility of continuing to match the first machine right up until the
second machine enters a final state. In other words it terminates the first
machine only when the second accepts. In the following example, finish-guarded
concatenation causes the move out of the machine that matches everything to be
delayed until the full end-of-input marker has been matched.

% GENERATE: finguard
% OPT: -p
% %%{
% machine finguard;
\begin{inline_code}
\begin{verbatim}
# Leave the catch-all machine on the last character of FIN.
main := any* :>> 'FIN';
\end{verbatim}
\end{inline_code}
% }%%
% END GENERATE

\begin{center}
\includegraphics[scale=0.55]{finguard}
\end{center}
\graphspace

Finish-guarded concatenation is equivalent to the following:

\verbspace
\begin{verbatim}
expr $(unique_name,0) . expr @(unique_name,1)
\end{verbatim}

\subsection{Left-Guarded Concatenation}

\verb|expr <: expr| 
\verbspace

This operator places
a higher priority on the left expression. It is useful if you want to prefix a
sequence with another sequence composed of some of the same characters. For
example, one can consume leading whitespace before tokenizing a sequence of
whitespace-separated words as in:

% GENERATE: leftguard
% OPT: -p
% %%{
% machine leftguard;
% action alpha {}
% action ws {}
% action start {}
% action fin {}
\begin{inline_code}
\begin{verbatim}
main := ( ' '* >start %fin ) <: ( ' ' $ws | [a-z] $alpha )*;
\end{verbatim}
\end{inline_code}
% }%%
% END GENERATE

\graphspace
\begin{center}
\includegraphics[scale=0.55]{leftguard}
\end{center}
\graphspace

Left-guarded concatenation is equivalent to the following:

\verbspace
\begin{verbatim}
expr $(unique_name,1) . expr >(unique_name,0)
\end{verbatim}
\verbspace

\subsection{Longest-Match Kleene Star}
\label{longest_match_kleene_star}

\verb|expr**| 
\verbspace

This version of kleene star puts a higher priority on staying in the
machine versus wrapping around and starting over. The LM kleene star is useful
when writing simple tokenizers.  These machines are built by applying the
longest-match kleene star to an alternation of token patterns, as in the
following.

\verbspace

% GENERATE: lmkleene
% OPT: -p
% %%{
% machine exfinpri;
% action A {}
% action B {}
\begin{inline_code}
\begin{verbatim}
# Repeat tokens, but make sure to get the longest match.
main := (
    lower ( lower | digit )* %A | 
    digit+ %B | 
    ' '
)**;
\end{verbatim}
\end{inline_code}
% }%%
% END GENERATE

\begin{center}
\includegraphics[scale=0.55]{lmkleene}
\end{center}
\graphspace

If a regular kleene star were used the machine above would not be able to
distinguish between extending a word and beginning a new one.  This operator is
equivalent to:

\verbspace
\begin{verbatim}
( expr $(unique_name,1) %(unique_name,0) )*
\end{verbatim}
\verbspace

When the kleene star is applied, transitions that go out of the machine and
back into it are made. These are assigned a priority of zero by the pending out
transition mechanism. This is less than the priority of one assigned to the
transitions leaving the final states but not leaving the machine. When two of
these transitions clash on the same character, the differing priorities cause
the transition that stays in the machine to take precedence.  The transition
that wraps around is dropped.

Note that this operator does not build a scanner in the traditional sense
because there is never any backtracking. To build a scanner in the traditional
sense use the Longest-Match machine construction described in Section
\ref{generating-scanners}.

\chapter{Interface to Host Program}

The Ragel code generator is very flexible. The generated code has no
dependencies and can be inserted in any function, perhaps inside a loop if so
desired.  The user is responsible for declaring and initializing a number of
required variables, including the current state and the pointer to the input
stream. These can live in any scope. Control of the input processing loop is
also possible: the user may break out of the processing loop and return to it
at any time.

In the case of C and D host languages, Ragel is able to generate very
fast-running code that implements state machines as directly executable code.
Since very large files strain the host language compiler, table-based code
generation is also supported. In the future we hope to provide a partitioned,
directly executable format that is able to reduce the burden on the host
compiler by splitting large machines across multiple functions.

In the case of Java and Ruby, table-based code generation is the only code
style supported. In the future this may be expanded to include other code
styles.

Ragel can be used to parse input in one block, or it can be used to parse input
in a sequence of blocks as it arrives from a file or socket.  Parsing the input
in a sequence of blocks brings with it a few responsibilities. If the parser
utilizes a scanner, care must be taken to not break the input stream anywhere
but token boundaries.  If pointers to the input stream are taken during
parsing, care must be taken to not use a pointer that has been invalidated by
movement to a subsequent block.  If the current input data pointer is moved
backwards it must not be moved past the beginning of the current block.

Figure \ref{basic-example} shows a simple Ragel program that does not have any
actions. The example tests the first argument of the program against a number
pattern and then prints the machine's acceptance status.

\begin{figure}
\small
\begin{verbatim}
#include <stdio.h>
#include <string.h>
%%{
    machine foo;
    write data;
}%%
int main( int argc, char **argv )
{
    int cs;
    if ( argc > 1 ) {
        char *p = argv[1];
        char *pe = p + strlen( p );
        %%{ 
            main := [0-9]+ ( '.' [0-9]+ )?;

            write init;
            write exec;
            write eof;
        }%%
    }
    printf("result = %i\n", cs >= foo_first_final );
    return 0;
}
\end{verbatim}
\caption{A basic Ragel example without any actions.}
\label{basic-example}
\end{figure}

\section{Variables Used by Ragel}

There are a number of variables which Ragel expects the user to declare. At a
very minimum the \verb|cs|, \verb|p| and \verb|pe| variables must be declared.
In Java and Ruby code the \verb|data| variable must also be declared. If
stack-based state machine control flow statements are used then the
\verb|stack| and \verb|top| variables are required. If a scanner is declared
then the \verb|act|, \verb|tokstart| and \verb|tokend| variables must be
declared.

\begin{itemize}

\item \verb|cs| - Current state. This must be an integer and it should persist
across invocations of the machine when the data is broken into blocks that are
processed independently.

\item \verb|p| - Data pointer. In C/D code this variable is expected to be a
pointer to the character data to process. It should be initialized to the
beginning of the data block on every run of the machine. In Java and Ruby it is
used as an offset to \verb|data| and must be an integer. In this case it should
be initialized to zero on every run of the machine.

\item \verb|pe| - Data end pointer. This should be initialized to \verb|p| plus
the data length on every run of the machine. In Java and Ruby code this should
be initialized to the data length.

\item \verb|data| - This variable is only required in Java and Ruby code. It
must be an array containting the data to process.

\item \verb|stack| - This must be an array of integers. It is used to store
integer values representing states. If the stack must resize dynamically the
Pre-push and Post-Pop statements can be used to do this (Sections
\ref{prepush} and \ref{postpop}).

\item \verb|top| - This must be an integer value and will be used as an offset
to \verb|stack|, giving the next available spot on the top of the stack.

\item \verb|act| - This must be an integer value. It is a variable sometimes
used by scanner code to keep track of the most recent successful pattern match.

\item \verb|tokstart| - This must be a pointer to character data. In Java and
Ruby code this must be an integer. See Section \ref{generating-scanners} for
more information.

\item \verb|tokend| - Also a pointer to character data.

\end{itemize}

\section{Alphtype Statement}

\begin{verbatim}
alphtype unsigned int;
\end{verbatim}
\verbspace

The alphtype statement specifies the alphabet data type that the machine
operates on. During the compilation of the machine, integer literals are expected to
be in the range of possible values of the alphtype.  Supported alphabet types
are \verb|char|, \verb|unsigned char|, \verb|short|, \verb|unsigned short|,
\verb|int|, \verb|unsigned int|, \verb|long|, and \verb|unsigned long|. 
The default is \verb|char|.

\section{Getkey Statement}

\begin{verbatim}
getkey fpc->id;
\end{verbatim}
\verbspace

Specify to Ragel how to retrieve the character that the machine operates on
from the pointer to the current element (\verb|p|). Any expression that returns
a value of the alphabet type
may be used. The getkey statement may be used for looking into element
structures or for translating the character to process. The getkey expression
defaults to \verb|(*p)|. In goto-driven machines the getkey expression may be
evaluated more than once per element processed, therefore it should not incur a
large cost nor preclude optimization.

\section{Access Statement}

\begin{verbatim}
access fsm->;
\end{verbatim}
\verbspace

The access statement allows one to tell Ragel how the generated code should
access the machine data that is persistent across processing buffer blocks.
This includes all variables except \verb|p| and \verb|pe|. This includes
\verb|cs|, \verb|top|, \verb|stack|, \verb|tokstart|, \verb|tokend| and \verb|act|.
This is useful if a machine is to be encapsulated inside a
structure in C code. The access statement can be used to give the name of
a pointer to the structure.

\section{Variable Statement}

\begin{verbatim}
variable p fsm->p;
\end{verbatim}
\verbspace

The variable statement allows one to tell ragel how to access a specific
variable. All of the variables that are declared by the user and
used by Ragel can be changed. This includes \verb|p|, \verb|pe|, \verb|cs|,
\verb|top|, \verb|stack|, \verb|tokstart|, \verb|tokend| and \verb|act|.
In Ruby and Java code generation the \verb|data| variable can also be changed.

\section{Pre-Push Statement}
\label{prepush}

\begin{verbatim}
prepush { 
	/* stack growing code */
}
\end{verbatim}
\verbspace

The prepush statement allows the user to supply stack management code that is
written out during the generation of fcall, immediately before the current
state is pushed to the stack. This statement can be used to test the number of
available spaces and dynamically grow the stack if necessary.

\section{Post-Pop Statement}
\label{postpop}

\begin{verbatim}
postpop { 
	/* stack shrinking code */
}
\end{verbatim}
\verbspace

The postpop statement allows the user to supply stack management code that is
written out during the generation of fret, immediately after the next state is
popped from the stack. This statement can be used to dynamically shrink the
stack.

\section{Write Statement}
\label{write-statement}

\begin{verbatim}
write <component> [options];
\end{verbatim}
\verbspace


The write statement is used to generate parts of the machine. 
There are four
components that can be generated by a write statement. These components are the
state machine's data, initialization code, execution code and EOF action
execution code. A write statement may appear before a machine is fully defined.
This allows one to write out the data first then later define the machine where
it is used. An example of this is shown in Figure \ref{fbreak-example}.

\subsection{Write Data}
\begin{verbatim}
write data [options];
\end{verbatim}
\verbspace

The write data statement causes Ragel to emit the constant static data needed
by the machine. In table-driven output styles (see Section \ref{genout}) this
is a collection of arrays that represent the states and transitions of the
machine.  In goto-driven machines much less data is emitted. At the very
minimum a start state \verb|name_start| is generated.  All variables written
out in machine data have both the \verb|static| and \verb|const| properties and
are prefixed with the name of the machine and an
underscore. The data can be placed inside a class, inside a function, or it can
be defined as global data.

Two variables are written that may be used to test the state of the machine
after a buffer block has been processed. The \verb|name_error| variable gives
the id of the state that the machine moves into when it cannot find a valid
transition to take. The machine immediately breaks out of the processing loop when
it finds itself in the error state. The error variable can be compared to the
current state to determine if the machine has failed to parse the input. If the
machine is complete, that is from every state there is a transition to a proper
state on every possible character of the alphabet, then no error state is required
and this variable will be set to -1.

The \verb|name_first_final| variable stores the id of the first final state. All of the
machine's states are sorted by their final state status before having their ids
assigned. Checking if the machine has accepted its input can then be done by
checking if the current state is greater-than or equal to the first final
state.

Data generation has several options:

\begin{itemize}
\setlength{\itemsep}{-2mm}
\item \verb|noerror  | - Do not generate the integer variable that gives the
id of the error state.
\item \verb|nofinal  | - Do not generate the integer variable that gives the
id of the first final state.
\item \verb|noprefix | - Do not prefix the variable names with the name of the
machine.
\end{itemize}

\subsection{Write Init}
\begin{verbatim}
write init;
\end{verbatim}
\verbspace

The write init statement causes Ragel to emit initialization code. This should
be executed once before the machine is started. At a very minimum this sets the
current state to the start state. If other variables are needed by the
generated code, such as call stack variables or scanner management
variables, they are also initialized here.

The \verb|nocs| option to the write init statement will cause ragel to skip
intialization of the cs variable. This is useful if the user wishes to use
custom logic to decide which state the specification should start in.

\subsection{Write Exec}
\begin{verbatim}
write exec [options];
\end{verbatim}
\verbspace

The write exec statement causes Ragel to emit the state machine's execution code.
Ragel expects several variables to be available to this code. At a very minimum, the
generated code needs access to the current character position \verb|p|, the ending
position \verb|pe| and the current state \verb|cs|, though \verb|pe|
can be excluded by specifying the \verb|noend| write option.
The \verb|p| variable is the cursor that the execute code will
used to traverse the input. The \verb|pe| variable should be set up to point to one
position past the last valid character in the buffer.

Other variables are needed when certain features are used. For example using
the \verb|fcall| or \verb|fret| statements requires \verb|stack| and
\verb|top| variables to be defined. If a longest-match construction is used,
variables for managing backtracking are required.

The write exec statement has one option. The \verb|noend| option tells Ragel
to generate code that ignores the end position \verb|pe|. In this
case the user must explicitly break out of the processing loop using
\verb|fbreak|, otherwise the machine will continue to process characters until
it moves into the error state. This option is useful if one wishes to process a
null terminated string. Rather than traverse the string to discover then length
before processing the input, the user can break out when the null character is
seen.  The example in Figure \ref{fbreak-example} shows the use of the
\verb|noend| write option and the \verb|fbreak| statement for processing a string.

\begin{figure}
\small
\begin{verbatim}
#include <stdio.h>
%% machine foo;
int main( int argc, char **argv )
{
    %% write data noerror nofinal;
    int cs, res = 0;
    if ( argc > 1 ) {
        char *p = argv[1];
        %%{ 
            main := 
                [a-z]+ 
                0 @{ res = 1; fbreak; };
            write init;
            write exec noend;
        }%%
    }
    printf("execute = %i\n", res );
    return 0;
}
\end{verbatim}
\caption{Use of {\tt noend} write option and the {\tt fbreak} statement for
processing a string.}
\label{fbreak-example}
\end{figure}


\subsection{Write EOF Actions}
\begin{verbatim}
write eof;
\end{verbatim}
\verbspace

The write EOF statement causes Ragel to emit code that executes EOF actions.
This write statement is only relevant if EOF actions have been embedded,
otherwise it does not generate anything. The EOF action code requires access to
the current state.

\subsection{Write Exports}
\label{export}

\begin{verbatim}
write exports;
\end{verbatim}
\verbspace

The export feature can be used to export simple machine definitions. Machine definitions
are marked for export using the \verb|export| keyword.

\verbspace
\begin{verbatim}
export machine_to_export = 0x44;
\end{verbatim}
\verbspace

When the write exports statement is used these machines are 
written out in the generated code. Defines are used for C and constant integers
are used for D, Java and Ruby. See Section \ref{import} for a description of the
import statement.
  
\section{Maintaining Pointers to Input Data}

In the creation of any parser it is not uncommon to require the collection of
the data being parsed.  It is always possible to collect data into a growable
buffer as the machine moves over it, however the copying of data is a somewhat
wasteful use of processor cycles. The most efficient way to collect data from
the parser is to set pointers into the input then later reference them.  This
poses a problem for uses of Ragel where the input data arrives in blocks, such
as over a socket or from a file. If a pointer is set in one buffer block but
must be used while parsing a following buffer block, some extra consideration
to correctness must be made.

The scanner constructions exhibit this problem, requiring the maintenance
code described in Section \ref{generating-scanners}. If a longest-match
construction has been used somewhere in the machine then it is possible to
take advantage of the required prefix maintenance code in the driver program to
ensure pointers to the input are always valid. If laying down a pointer one can
set \verb|tokstart| at the same spot or ahead of it. When data is shifted in
between loops the user must also shift the pointer.  In this way it is possible
to maintain pointers to the input that will always be consistent.

\begin{figure}
\small
\begin{verbatim}
    int have = 0;
    while ( 1 ) {
        char *p, *pe, *data = buf + have;
        int len, space = BUFSIZE - have;

        if ( space == 0 ) { 
            fprintf(stderr, "BUFFER OUT OF SPACE\n");
            exit(1);
        }

        len = fread( data, 1, space, stdin );
        if ( len == 0 )
            break;

        /* Find the last newline by searching backwards. */
        p = buf;
        pe = data + len - 1;
        while ( *pe != '\n' && pe >= buf )
            pe--;
        pe += 1;

        %% write exec;

        /* How much is still in the buffer? */
        have = data + len - pe;
        if ( have > 0 )
            memmove( buf, pe, have );

        if ( len < space )
            break;
    }
\end{verbatim}
\caption{An example of line-oriented processing.}
\label{line-oriented}
\end{figure}

In general, there are two approaches for guaranteeing the consistency of
pointers to input data. The first approach is the one just described;
lay down a marker from an action,
then later ensure that the data the marker points to is preserved ahead of
the buffer on the next execute invocation. This approach is good because it
allows the parser to decide on the pointer-use boundaries, which can be
arbitrarily complex parsing conditions. A downside is that it requires any
pointers that are set to be corrected in between execute invocations.

The alternative is to find the pointer-use boundaries before invoking the execute
routine, then pass in the data using these boundaries. For example, if the
program must perform line-oriented processing, the user can scan backwards from
the end of an input block that has just been read in and process only up to the
first found newline. On the next input read, the new data is placed after the
partially read line and processing continues from the beginning of the line.
An example of line-oriented processing is given in Figure \ref{line-oriented}.


\section{Running the Executables}

Ragel is broken down into two parts: a frontend that compiles machines
and emits them in an XML format, and a backend that generates code or a
Graphviz Dot file from the XML data. The purpose of the XML-based intermediate
format is to allow users to inspect their compiled state machines and to
interface Ragel to other tools such as custom visualizers, code generators or
analysis tools. The split also serves to reduce the complexity of the Ragel
program by strictly separating the data structures and algorithms that are used
to compile machines from those that are used to generate code. 

\vspace{10pt}

\noindent The frontend program is called \verb|ragel|. It takes as an argument the host
language. This can be:

\begin{itemize}
\item \verb|-C  | for C/C++/Objective-C code (default)
\item \verb|-D  | for D code.
\item \verb|-J  | for Java code.
\item \verb|-R  | for Ruby code.
\end{itemize}

\noindent There are four code backend programs. These are:

\begin{itemize}
\item \verb|rlgen-cd    | generate code for the C-based and D languages.
\item \verb|rlgen-java  | generate code for the Java language.
\item \verb|rlgen-ruby  | generate code for the Ruby language.
\item \verb|rlgen-dot   | generate a Graphviz Dot file.
\end{itemize}

\section{Choosing a Generated Code Style (C/D only)}
\label{genout}

There are three styles of code output to choose from. Code style affects the
size and speed of the compiled binary. Changing code style does not require any
change to the Ragel program. There are two table-driven formats and a goto
driven format.

In addition to choosing a style to emit, there are various levels of action
code reuse to choose from.  The maximum reuse levels (\verb|-T0|, \verb|-F0|
and \verb|-G0|) ensure that no FSM action code is ever duplicated by encoding
each transition's action list as static data and iterating
through the lists on every transition. This will normally result in a smaller
binary. The less action reuse options (\verb|-T1|, \verb|-F1| and \verb|-G1|)
will usually produce faster running code by expanding each transition's action
list into a single block of code, eliminating the need to iterate through the
lists. This duplicates action code instead of generating the logic necessary
for reuse. Consequently the binary will be larger. However, this tradeoff applies to
machines with moderate to dense action lists only. If a machine's transitions
frequently have less than two actions then the less reuse options will actually
produce both a smaller and a faster running binary due to less action sharing
overhead. The best way to choose the appropriate code style for your
application is to perform your own tests.

The table-driven FSM represents the state machine as constant static data. There are
tables of states, transitions, indices and actions. The current state is
stored in a variable. The execution is simply a loop that looks up the current
state, looks up the transition to take, executes any actions and moves to the
target state. In general, the table-driven FSM can handle any machine, produces
a smaller binary and requires a less expensive host language compile, but
results in slower running code.  Since the table-driven format is the most
flexible it is the default code style.

The flat table-driven machine is a table-based machine that is optimized for
small alphabets. Where the regular table machine uses the current character as
the key in a binary search for the transition to take, the flat table machine
uses the current character as an index into an array of transitions. This is
faster in general, however is only suitable if the span of possible characters
is small.

The goto-driven FSM represents the state machine using goto and switch
statements. The execution is a flat code block where the transition to take is
computed using switch statements and directly executable binary searches.  In
general, the goto FSM produces faster code but results in a larger binary and a
more expensive host language compile.

The goto-driven format has an additional action reuse level (\verb|-G2|) that
writes actions directly into the state transitioning logic rather than putting
all the actions together into a single switch. Generally this produces faster
running code because it allows the machine to encode the current state using
the processor's instruction pointer. Again, sparse machines may actually
compile to smaller binaries when \verb|-G2| is used due to less state and
action management overhead. For many parsing applications \verb|-G2| is the
preferred output format.

\verbspace
\begin{center}
\begin{tabular}{|c|c|}
\hline
\multicolumn{2}{|c|}{\bf Code Output Style Options} \\
\hline
\verb|-T0|&binary search table-driven\\
\hline
\verb|-T1|&binary search, expanded actions\\
\hline
\verb|-F0|&flat table-driven\\
\hline
\verb|-F1|&flat table, expanded actions\\
\hline
\verb|-G0|&goto-driven\\
\hline
\verb|-G1|&goto, expanded actions\\
\hline
\verb|-G2|&goto, in-place actions\\
\hline
\end{tabular}
\end{center}

\chapter{Beyond the Basic Model}

\section{Parser Modularization}

It is possible to use Ragel's machine construction and action embedding
operators to specify an entire parser using a single regular expression. In
many cases this is the desired way to specify a parser in Ragel. However, in
some scenarios, the language to parse may be so large that it is difficult to
think about it as a single regular expression. It may shift between distinct
parsing strategies, in which case modularization into several coherent blocks
of the language may be appropriate.

It may also be the case that patterns that compile to a large number of states
must be used in a number of different contexts and referencing them in each
context results in a very large state machine. In this case, an ability to reuse
parsers would reduce code size.

To address this, distinct regular expressions may be instantiated and linked
together by means of a jumping and calling mechanism. This mechanism is
analogous to the jumping to and calling of processor instructions. A jump
command, given in action code, causes control to be immediately passed to
another portion of the machine by way of setting the current state variable. A
call command causes the target state of the current transition to be pushed to
a state stack before control is transferred.  Later on, the original location
may be returned to with a return statement. In the following example, distinct
state machines are used to handle the parsing of two types of headers.

% GENERATE: call
% %%{
% 	machine call;
\begin{inline_code}
\begin{verbatim}
action return { fret; }
action call_date { fcall date; }
action call_name { fcall name; }

# A parser for date strings.
date := [0-9][0-9] '/' 
        [0-9][0-9] '/' 
        [0-9][0-9][0-9][0-9] '\n' @return;

# A parser for name strings.
name := ( [a-zA-Z]+ | ' ' )** '\n' @return;

# The main parser.
headers = 
    ( 'from' | 'to' ) ':' @call_name | 
    ( 'departed' | 'arrived' ) ':' @call_date;

main := headers*;
\end{verbatim}
\end{inline_code}
% }%%
% %% write data;
% void f()
% {
% 	%% write init;
% 	%% write exec;
% }
% END GENERATE

Calling and jumping should be used carefully as they are operations that take
one out of the domain of regular languages. A machine that contains a call or
jump statement in one of its actions should be used as an argument to a machine
construction operator only with considerable care. Since DFA transitions may
actually represent several NFA transitions, a call or jump embedded in one
machine can inadvertently terminate another machine that it shares prefixes
with. Despite this danger, theses statements have proven useful for tying
together sub-parsers of a language into a parser for the full language,
especially for the purpose of modularizing code and reducing the number of
states when the machine contains frequently recurring patterns.

Section \ref{vals} describes the jump and call statements that are used to
transfer control. These statements make use of two variables that must be
declared by the user, \verb|stack| and \verb|top|. The \verb|stack| variable
must be an array of integers and \verb|top| must be a single integer, which
will point to the next available space in \verb|stack|. Sections \ref{prepush}
and \ref{postpop} describe the Pre-Push and Post-Pop statements which can be
used to implement a dynamically resizable array.

\section{Referencing Names}
\label{labels}

This section describes how to reference names in epsilon transitions and
action-based control-flow statements such as \verb|fgoto|. There is a hierarchy
of names implied in a Ragel specification.  At the top level are the machine
instantiations. Beneath the instantiations are labels and references to machine
definitions. Beneath those are more labels and references to definitions, and
so on.

Any name reference may contain multiple components separated with the \verb|::|
compound symbol.  The search for the first component of a name reference is
rooted at the join expression that the epsilon transition or action embedding
is contained in. If the name reference is not contained in a join,
the search is rooted at the machine definition that the epsilon transition or
action embedding is contained in. Each component after the first is searched
for beginning at the location in the name tree that the previous reference
component refers to.

In the case of action-based references, if the action is embedded more than
once, the local search is performed for each embedding and the result is the
union of all the searches. If no result is found for action-based references then
the search is repeated at the root of the name tree.  Any action-based name
search may be forced into a strictly global search by prefixing the name
reference with \verb|::|.

The final component of the name reference must resolve to a unique entry point.
If a name is unique in the entire name tree it can be referenced as is. If it
is not unique it can be specified by qualifying it with names above it in the
name tree. However, it can always be renamed.

% FIXME: Should fit this in somewhere.
% Some kinds of name references are illegal. Cannot call into longest-match
% machine, can only call its start state. Cannot make a call to anywhere from
% any part of a longest-match machine except a rule's action. This would result
% in an eventual return to some point inside a longest-match other than the
% start state. This is banned for the same reason a call into the LM machine is
% banned.


\section{Scanners}
\label{generating-scanners}

Scanners are very much intertwined with regular-languages and their
corresponding processors. For this reason Ragel supports the definition of
Scanners.  The generated code will repeatedly attempt to match patterns from a
list, favouring longer patterns over shorter patterns.  In the case of
equal-length matches, the generated code will favour patterns that appear ahead
of others. When a scanner makes a match it executes the user code associated
with the match, consumes the input then resumes scanning.

\verbspace
\begin{verbatim}
<machine_name> := |* 
        pattern1 => action1;
        pattern2 => action2;
        ...
    *|;
\end{verbatim}
\verbspace

On the surface, Ragel scanners are similar to those defined by Lex. Though
there is a key distinguishing feature: patterns may be arbitrary Ragel
expressions and can therefore contain embedded code. With a Ragel-based scanner
the user need not wait until the end of a pattern before user code can be
executed.

Scanners can be used to process sub-languages, as well as for tokenizing
programming languages. In the following example a scanner is used to tokenize
the contents of a header field.

\begin{inline_code}
\begin{verbatim}
word = [a-z]+;
head_name = 'Header';

header := |*
    word;
    ' ';
    '\n' => { fret; };
*|;

main := ( head_name ':' @{ fcall header; } )*;
\end{verbatim}
\end{inline_code}
\verbspace

The scanner construction has a purpose similar to the longest-match kleene star
operator \verb|**|. The key
difference is that a scanner is able to backtrack to match a previously matched
shorter string when the pursuit of a longer string fails.  For this reason the
scanner construction operator is not a pure state machine construction
operator. It relies on several variables which enable it to backtrack and make
pointers to the matched input text available to the user.  For this reason
scanners must be immediately instantiated. They cannot be defined inline or
referenced by another expression. Scanners must be jumped to or called.

Scanners rely on the \verb|tokstart|, \verb|tokend| and \verb|act|
variables to be present so that it can backtrack and make pointers to the
matched text available to the user. If input is processed using multiple calls
to the execute code then the user must ensure that when a token is only
partially matched that the prefix is preserved on the subsequent invocation of
the execute code.

The \verb|tokstart| variable must be defined as a pointer to the input data.
It is used for recording where the current token match begins. This variable
may be used in action code for retrieving the text of the current match.  Ragel
ensures that in between tokens and outside of the longest-match machines that
this pointer is set to null. In between calls to the execute code the user must
check if \verb|tokstart| is set and if so, ensure that the data it points to is
preserved ahead of the next buffer block. This is described in more detail
below.

The \verb|tokend| variable must also be defined as a pointer to the input data.
It is used for recording where a match ends and where scanning of the next
token should begin. This can also be used in action code for retrieving the
text of the current match.

The \verb|act| variable must be defined as an integer type. It is used for
recording the identity of the last pattern matched when the scanner must go
past a matched pattern in an attempt to make a longer match. If the longer
match fails it may need to consult the \verb|act| variable. In some cases, use 
of the \verb|act|
variable can be avoided because the value of the current state is enough
information to determine which token to accept, however in other cases this is
not enough and so the \verb|act| variable is used. 

When the longest-match operator is in use, the user's driver code must take on
some buffer management functions. The following algorithm gives an overview of
the steps that should be taken to properly use the longest-match operator.

\begin{itemize}
\setlength{\parskip}{0pt}
\item Read a block of input data.
\item Run the execute code.
\item If \verb|tokstart| is set, the execute code will expect the incomplete
token to be preserved ahead of the buffer on the next invocation of the execute
code.  
\begin{itemize}
\item Shift the data beginning at \verb|tokstart| and ending at \verb|pe| to the
beginning of the input buffer.
\item Reset \verb|tokstart| to the beginning of the buffer. 
\item Shift \verb|tokend| by the distance from the old value of \verb|tokstart|
to the new value. The \verb|tokend| variable may or may not be valid.  There is
no way to know if it holds a meaningful value because it is not kept at null
when it is not in use. It can be shifted regardless.
\end{itemize}
\item Read another block of data into the buffer, immediately following any
preserved data.
\item Run the scanner on the new data.
\end{itemize}

Figure \ref{preserve_example} shows the required handling of an input stream in
which a token is broken by the input block boundaries. After processing up to
and including the ``t'' of ``characters'', the prefix of the string token must be
retained and processing should resume at the ``e'' on the next iteration of
the execute code.

If one uses a large input buffer for collecting input then the number of times
the shifting must be done will be small. Furthermore, if one takes care not to
define tokens that are allowed to be very long and instead processes these
items using pure state machines or sub-scanners, then only a small amount of
data will ever need to be shifted.

\begin{figure}
\begin{verbatim}
      a)           A stream "of characters" to be scanned.
                   |        |          |
                   p        tokstart   pe

      b)           "of characters" to be scanned.
                   |          |        |
                   tokstart   p        pe
\end{verbatim}
\caption{Following an invocation of the execute code there may be a partially
matched token (a). The data of the partially matched token 
must be preserved ahead of the new data on the next invocation (b).}
\label{preserve_example}
\end{figure}

Since scanners attempt to make the longest possible match of input, in some
cases they are not able to identify a token upon parsing its final character,
they must wait for a lookahead character. For example if trying to match words,
the token match must be triggered on following whitespace in case more
characters of the word have yet to come. The user must therefore arrange for an
EOF character to be sent to the scanner to flush out any token that has not yet
been matched.  The user can exclude a single character from the entire scanner
and use this character as the EOF character, possibly specifying an EOF action.
For most scanners, zero is a suitable choice for the EOF character. 

Alternatively, if whitespace is not significant and ignored by the scanner, the
final real token can be flushed out by simply sending an additional whitespace
character on the end of the stream. If the real stream ends with whitespace
then it will simply be extended and ignored. If it does not, then the last real token is
guaranteed to be flushed and the dummy EOF whitespace ignored.
An example scanner processing loop is given in Figure \ref{scanner-loop}.

\begin{figure}
\small
\begin{verbatim}
    int have = 0;
    bool done = false;
    while ( !done ) {
        /* How much space is in the buffer? */
        int space = BUFSIZE - have;
        if ( space == 0 ) {
            /* Buffer is full. */
            cerr << "TOKEN TOO BIG" << endl;
            exit(1);
        }

        /* Read in a block after any data we already have. */
        char *p = inbuf + have;
        cin.read( p, space );
        int len = cin.gcount();

        /* If no data was read, send the EOF character. */
        if ( len == 0 ) {
            p[0] = 0, len++;
            done = true;
        }

        char *pe = p + len;
        %% write exec;

        if ( cs == RagelScan_error ) {
            /* Machine failed before finding a token. */
            cerr << "PARSE ERROR" << endl;
            exit(1);
        }

        if ( tokstart == 0 )
            have = 0;
        else {
            /* There is a prefix to preserve, shift it over. */
            have = pe - tokstart;
            memmove( inbuf, tokstart, have );
            tokend = inbuf + (tokend-tokstart);
            tokstart = inbuf;
        }
    }
\end{verbatim}
\caption{A processing loop for a scanner.}
\label{scanner-loop}
\end{figure}

\section{State Charts}

In addition to supporting the construction of state machines using regular
languages, Ragel provides a way to manually specify state machines using
state charts.  The comma operator combines machines together without any
implied transitions. The user can then manually link machines by specifying
epsilon transitions with the \verb|->| operator.  Epsilon transitions are drawn
between the final states of a machine and entry points defined by labels.  This
makes it possible to build machines using the explicit state-chart method while
making minimal changes to the Ragel language. 

An interesting feature of Ragel's state chart construction method is that it
can be mixed freely with regular expression constructions. A state chart may be
referenced from within a regular expression, or a regular expression may be
used in the definition of a state chart transition.

\subsection{Join}

\verb|expr , expr , ...|
\verbspace

Join a list of machines together without
drawing any transitions, without setting up a start state, and without
designating any final states. Transitions between the machines may be specified
using labels and epsilon transitions. The start state must be explicity
specified with the ``start'' label. Final states may be specified with an
epsilon transition to the implicitly created ``final'' state. The join
operation allows one to build machines using a state chart model.

\subsection{Label}

\verb|label: expr| 
\verbspace

Attaches a label to an expression. Labels can be
used as the target of epsilon transitions and explicit control transfer
statements such as \verb|fgoto| and \verb|fnext| in action
code.

\subsection{Epsilon}

\verb|expr -> label| 
\verbspace

Draws an epsilon transition to the state defined
by \verb|label|.  Epsilon transitions are made deterministic when join
operators are evaluated. Epsilon transitions that are not in a join operation
are made deterministic when the machine definition that contains the epsilon is
complete. See Section \ref{labels} for information on referencing labels.

\subsection{Simplifying State Charts}

There are two benefits to providing state charts in Ragel. The first is that it
allows us to take a state chart with a full listing of states and transitions
and simplify it in selective places using regular expressions.

The state chart method of specifying parsers is very common.  It is an
effective programming technique for producing robust code. The key disadvantage
becomes clear when one attempts to comprehend a large parser specified in this
way.  These programs usually require many lines, causing logic to be spread out
over large distances in the source file. Remembering the function of a large
number of states can be difficult and organizing the parser in a sensible way
requires discipline because branches and repetition present many file layout
options.  This kind of programming takes a specification with inherent
structure such as looping, alternation and concatenation and expresses it in a
flat form. 

If we could take an isolated component of a manually programmed state chart,
that is, a subset of states that has only one entry point, and implement it
using regular language operators then we could eliminate all the explicit
naming of the states contained in it. By eliminating explicitly named states
and replacing them with higher-level specifications we simplify a state machine
specification.

For example, sometimes chains of states are needed, with only a small number of
possible characters appearing along the chain. These can easily be replaced
with a concatenation of characters. Sometimes a group of common states
implement a loop back to another single portion of the machine. Rather than
manually duplicate all the transitions that loop back, we may be able to
express the loop using a kleene star operator.

Ragel allows one to take this state map simplification approach. We can build
state machines using a state map model and implement portions of the state map
using regular languages. In place of any transition in the state machine,
entire sub-state machines can be given. These can encapsulate functionality
defined elsewhere. An important aspect of the Ragel approach is that when we
wrap up a collection of states using a regular expression we do not lose
access to the states and transitions. We can still execute code on the
transitions that we have encapsulated.

\subsection{Dropping Down One Level of Abstraction}
\label{down}

The second benefit of incorporating state charts into Ragel is that it permits
us to bypass the regular language abstraction if we need to. Ragel's action
embedding operators are sometimes insufficient for expressing certain parsing
tasks.  In the same way that is useful for C language programmers to drop down
to assembly language programming using embedded assembler, it is sometimes
useful for the Ragel programmer to drop down to programming with state charts.

In the following example, we wish to buffer the characters of an XML CDATA
sequence. The sequence is terminated by the string \verb|]]>|. The challenge
in our application is that we do not wish the terminating characters to be
buffered. An expression of the form \verb|any* @buffer :>> ']]>'| will not work
because the buffer will always contain the characters \verb|]]| on the end.
Instead, what we need is to delay the buffering of \hspace{0.25mm} \verb|]|
characters until a time when we
abandon the terminating sequence and go back into the main loop. There is no
easy way to express this using Ragel's regular expression and action embedding
operators, and so an ability to drop down to the state chart method is useful.

% GENERATE: dropdown
% OPT: -p
% %%{
% machine dropdown;
\begin{inline_code}
\begin{verbatim}
action bchar { buff( fpc ); }    # Buffer the current character.
action bbrack1 { buff( "]" ); }
action bbrack2 { buff( "]]" ); }

CDATA_body =
start: (
     ']' -> one |
     (any-']') @bchar ->start
),
one: (
     ']' -> two |
     [^\]] @bbrack1 @bchar ->start
),
two: (
     '>' -> final |
     ']' @bbrack1 -> two |
     [^>\]] @bbrack2 @bchar ->start
);
\end{verbatim}
\end{inline_code}
% main := CDATA_body;
% }%%
% END GENERATE

\graphspace
\begin{center}
\includegraphics[scale=0.55]{dropdown}
\end{center}


\section{Semantic Conditions}
\label{semantic}

Many communication protocols contain variable-length fields, where the length
of the field is given ahead of the field as a value. This
problem cannot be expressed using regular languages because of its
context-dependent nature. The prevalence of variable-length fields in
communication protocols motivated us to introduce semantic conditions into
the Ragel language.

A semantic condition is a block of user code that is executed immediately
before a transition is taken. If the code returns a value of true, the
transition may be taken.  We can now embed code that extracts the length of a
field, then proceed to match $n$ data values.

% GENERATE: conds1
% OPT: -p
% %%{
% machine conds1;
% number = digit+;
\begin{inline_code}
\begin{verbatim}
action rec_num { i = 0; n = getnumber(); }
action test_len { i++ < n }
data_fields = (
    'd' 
    [0-9]+ %rec_num 
    ':'
    ( [a-z] when test_len )*
)**;
\end{verbatim}
\end{inline_code}
% main := data_fields;
% }%%
% END GENERATE

\begin{center}
\includegraphics[scale=0.55]{conds1}
\end{center}
\graphspace

The Ragel implementation of semantic conditions does not force us to give up the
compositional property of Ragel definitions. For example, a machine that tests
the length of a field using conditions can be unioned with another machine
that accepts some of the same strings, without the two machines interfering with
one another. The user need not be concerned about whether or not the result of the
semantic condition will affect the matching of the second machine.

To see this, first consider that when a user associates a condition with an
existing transition, the transition's label is translated from the base character
to its corresponding value in the space that represents ``condition $c$ true''. Should
the determinization process combine a state that has a conditional transition
with another state that has a transition on the same input character but
without a condition, then the condition-less transition first has its label
translated into two values, one to its corresponding value in the space that
represents ``condition $c$ true'' and another to its corresponding value in the
space that represents ``condition $c$ false''. It
is then safe to combine the two transitions. This is shown in the following
example.  Two intersecting patterns are unioned, one with a condition and one
without. The condition embedded in the first pattern does not affect the second
pattern.

% GENERATE: conds2
% OPT: -p
% %%{
% machine conds2;
% number = digit+;
\begin{inline_code}
\begin{verbatim}
action test_len { i++ < n }
action one { /* accept pattern one */ }
action two { /* accept pattern two */ }
patterns = 
    ( [a-z] when test_len )+ %one |
    [a-z][a-z0-9]* %two;
main := patterns '\n';
\end{verbatim}
\end{inline_code}
% }%%
% END GENERATE

\begin{center}
\includegraphics[scale=0.55]{conds2}
\end{center}
\graphspace

There are many more potential uses for semantic conditions. The user is free to
use arbitrary code and may therefore perform actions such as looking up names
in dictionaries, validating input using external parsing mechanisms or
performing checks on the semantic structure of input seen so far. In the
next section we describe how Ragel accommodates several common parser
engineering problems.

\vspace{10pt}

\noindent {\large\bf Note:} The semantic condition feature works only with
alphabet types that are smaller in width than the \verb|long| type. To
implement semantic conditions Ragel needs to be able to allocate characters
from the alphabet space. Ragel uses these allocated characters to express
"character C with condition P true" or "C with P false." Since internally Ragel
uses longs to store characters there is no room left in the alphabet space
unless an alphabet type smaller than long is used.

\section{Implementing Lookahead}

There are a few strategies for implementing lookahead in Ragel programs.
Pending out actions, which are described in Section \ref{out-actions}, can be
used as a form of lookahead.  Ragel also provides the \verb|fhold| directive
which can be used in actions to prevent the machine from advancing over the
current character. It is also possible to manually adjust the current character
position by shifting it backwards using \verb|fexec|, however when this is
done, care must be taken not to overstep the beginning of the current buffer
block. In both the use of \verb|fhold| and \verb|fexec| the user must be
cautious of combining the resulting machine with another in such a way that the
transition on which the current position is adjusted is not combined with a
transition from the other machine.

\section{Parsing Recursive Language Structures}

In general Ragel cannot handle recursive structures because the grammar is
interpreted as a regular language. However, depending on what needs to be
parsed it is sometimes practical to implement the recursive parts using manual
coding techniques. This often works in cases where the recursive structures are
simple and easy to recognize, such as in the balancing of parentheses

One approach to parsing recursive structures is to use actions that increment
and decrement counters or otherwise recognize the entry to and exit from
recursive structures and then jump to the appropriate machine defnition using
\verb|fcall| and \verb|fret|. Alternatively, semantic conditions can be used to
test counter variables.

A more traditional approach is to call a separate parsing function (expressed
in the host language) when a recursive structure is entered, then later return
when the end is recognized.

\end{document}
